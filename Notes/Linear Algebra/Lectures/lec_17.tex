\section{Applications of Eigenvalues and Eigenvectors}
\begin{exercise}
    Solve the linear differential equation \(y^{\prime} =6y\).
\end{exercise}
\begin{solution}
    Trivial. \(y=Ce^{6x} \) for \(C \) some constant.
\end{solution}
\begin{definition}[System of First-Order Linear Differential Equations]
    A system of first order linear differential equations is of the form
    \[
        \left\{
            \begin{array}{ccc}
                y_1^{\prime} &=&a_{11}y_1 +a_{12}y_{2}+\cdots+a_{1n}y_n\\
                y_2^{\prime} &=&a_{21}y_1 +a_{22}y_2 +\cdots+a_{2n}y_n\\
                &\vdots&\\
                y_n^{\prime} &=&a_{n1}y_1 +a_{n2}y_2 +\cdots+a_{nn}y_n          
            \end{array}
        \right.
    \]
\end{definition}
We can write the above definiton as
\[
    \mathbf{y} ^{\prime} =\begin{bmatrix}
        y_1^{\prime}   \\
          y_2^{\prime} \\
          \vdots\\
          y_n^{\prime}  \\
    \end{bmatrix}\quad \mathbf{y} =\begin{bmatrix}
         y_1 \\
          y_2\\
          \vdots\\
          y_n\\
    \end{bmatrix}\quad A=\begin{bmatrix}
        a_{11}  &a_{12}   &\cdots  &a_{1n}    \\
         a_{21} &a_{22}   &\cdots  &a_{2n}    \\
         \vdots&\vdots  &\ddots  &\vdots   \\
         a_{n1} &a_{n2}   &\cdots  &a_{nn}    \\
    \end{bmatrix}
\]
Suppose for simplicity that \(A\) is diagonal.
\[
    \begin{bmatrix}
         y^{\prime} _1 \\
          y^{\prime} _2\\
          \vdots\\
          y^{\prime} _n\\
    \end{bmatrix} = \begin{bmatrix}
        a_{11}  & 0 &0  &0   \\
         0&a_{22}  &0  &0   \\
         0&  0&\ddots  &0   \\
         0&  0&  0&   a_{nn} \\
    \end{bmatrix} \begin{bmatrix}
         y_1 \\
          y_2\\
          \vdots\\
          y_n\\
    \end{bmatrix}
\]
Then obviously, \(y^{\prime}_i =a_{ii}y_1 \).
\begin{exercise}
    Solve the system of first order differential equations.
    \[
        y_1^{\prime} =2y_1
    \]
    \[
        y_2^{\prime} =3y_2
    \]
\end{exercise}
\begin{solution}
    \[
        \begin{bmatrix}
             y_1^{\prime}  \\
              y_2^{\prime} \\
        \end{bmatrix} = \begin{bmatrix}
            2 &0   \\
             0&3   \\
        \end{bmatrix} \begin{bmatrix}
             y_1 \\
             y_2 \\
        \end{bmatrix}
    \]
    Therefore, \(y_1 = C_1e^{2t} \) and \(y_2 =C_2e^{3t} \).
\end{solution}
\begin{exercise}
    Solve the system of first-order differential equations.
    \[
        y_1^{\prime} =-y_1
    \]
    \[
        y_2^{\prime} =6y_2
    \]
    \[
        y_3^{\prime} =y_3
    \]
\end{exercise}
\begin{solution}
\[
    \begin{bmatrix}
         y_1^{\prime}  \\
          y_2^{\prime} \\
          y_3^{\prime} \\
    \end{bmatrix} = \begin{bmatrix}
        -1 &0  &0   \\
         0&6  &0   \\
         0&0  &1   \\
    \end{bmatrix}\begin{bmatrix}
        y_1  \\
          y_2\\
          y_3\\
    \end{bmatrix}
\]
\[
    \therefore \begin{bmatrix}
         y_1 \\
          y_2\\
          y_3\\
    \end{bmatrix}=\begin{bmatrix}
         C_1e^{-t}  \\
          C_2e^{6t} \\
          C_3e^t\\
    \end{bmatrix}
\]
\end{solution}
\begin{remark}
    This in and of itself is not a strong application, it's just remembering a form for the solution to a differential equation. However, we are laying the groundwork for a stronger application. What happens if \(A\) is not diagonal, but \textbf{diagonalizable}?
\end{remark}
\begin{exercise}
    Solve the system of first-order differential equations.
    \[
        y_1^{\prime} =y_1 +3y_2
    \]
    \[
        y_2^{\prime} =2y_2
    \]
\end{exercise}
\begin{longsolution}
\[
    \begin{bmatrix}
         y_1^{\prime}  \\
          y_2^{\prime} \\
    \end{bmatrix} = \begin{bmatrix}
        1 &3   \\
         0&2   \\
    \end{bmatrix} \begin{bmatrix}
         y_1 \\
          y_2\\
    \end{bmatrix}
\]
We know that the coefficient matrix \(A\) is diagonalizable since it has \(2\) distinct eigenvalues.
\[
    \begin{bmatrix}
        0 &-3   \\
         0& -1  \\
    \end{bmatrix} \begin{bmatrix}
         x \\
         y \\
    \end{bmatrix}= \mathbf{0} \Longrightarrow \begin{bmatrix}
        -3y    \\
        -y    \\
    \end{bmatrix} = \begin{bmatrix}
         0 \\
         0 \\
    \end{bmatrix}
\]
For \(\lambda =1\), \(y=0\) and \(x\in\mathbb{R} \). Thus we have the eigenvector \((1,0)\). Now take \(\lambda =2\).
\[
    \begin{bmatrix}
        1 &-3   \\
         0&  0 \\
    \end{bmatrix} \begin{bmatrix}
         x \\
         y \\
    \end{bmatrix}= \mathbf{0} \Longrightarrow \begin{bmatrix}
         x -3y \\
          0\\
    \end{bmatrix}=\begin{bmatrix}
         0 \\
         0 \\
    \end{bmatrix}
\]
Therefore \(x=3y\), and the eigenspace is spanned by \((3,1)\). Thus we have the eigenvectors \((1,0),(3,1)\) and build the matrix
\[
    P\coloneqq \begin{bmatrix}
        1 &3   \\
        0 &1   \\
    \end{bmatrix}
\]
with inverse
\[
    \begin{bmatrix}
        1 &-3   \\
         0&  1 \\
    \end{bmatrix}
\]
It follows that
\[
    \begin{bmatrix}
        1 &3   \\
        0 &1   \\
    \end{bmatrix} \begin{bmatrix}
        1 &3   \\
         0&2   \\
    \end{bmatrix} \begin{bmatrix}
        1 &-3   \\
        0 &1   \\
    \end{bmatrix}= \begin{bmatrix}
        1 &0   \\
         0&2   \\
    \end{bmatrix}
\]
Define this diagonal matrix \(D\) and observe
\[
    \mathbf{y}^{\prime} =PDP ^{-1} \mathbf{y} 
\]
Define \(\mathbf{w} \coloneqq P ^{-1} \mathbf{y} \). It follows that
\[
    \mathbf{y} ^{\prime} =PD \mathbf{w}
\]
Now notice that
\[
    \mathbf{w} = \begin{bmatrix}
         w_1 \\
         w_2 \\
    \end{bmatrix}= \begin{bmatrix}
        1 &-3   \\
         0&1   \\
    \end{bmatrix} \begin{bmatrix}
         y_1 \\
          y_2\\
    \end{bmatrix} = \begin{bmatrix}
        y_1 - 3y_2  \\
         y_2   \\
    \end{bmatrix} \Longrightarrow \begin{bmatrix}
         w_1^{\prime}  \\
          w_2^{\prime} \\
    \end{bmatrix} = \begin{bmatrix}
         y_1^{\prime} -3y_2^{\prime}  \\
          y_2^{\prime} \\
    \end{bmatrix} = \begin{bmatrix}
        1 &-3   \\
         0&1   \\
    \end{bmatrix} \begin{bmatrix}
         y_1^{\prime}  \\
          y_2^{\prime} \\
    \end{bmatrix}
\]
Thus we have \(\mathbf{w} ^{\prime}  = P ^{-1} \mathbf{y} ^{\prime} \Longrightarrow \mathbf{y} ^{\prime} =P \mathbf{w}^{\prime} \). We thus have
\[
    PD \mathbf{w} =P \mathbf{w} ^{\prime}  \Longrightarrow D \mathbf{w} = \mathbf{w} ^{\prime} 
\]
\[
    \Longrightarrow \begin{bmatrix}
         w_1^{\prime}  \\
          w_2^{\prime} \\
    \end{bmatrix} = \begin{bmatrix}
        1 &0   \\
         0&2   \\
    \end{bmatrix} \begin{bmatrix}
         w_1 \\
          w_2\\
    \end{bmatrix}
\]
\[
    \therefore \begin{bmatrix}
         w_1 \\
          w_2\\
    \end{bmatrix}=\begin{bmatrix}
         C_1 e^t \\
          C_2 e^{2t} \\
    \end{bmatrix}
\]
Recall \(\mathbf{w} =P ^{-1}  \mathbf{y} \). We thus have \(P \mathbf{w} =\mathbf{y}\):
\[
    \begin{bmatrix}
        1 &3   \\
         0&1   \\
    \end{bmatrix} \begin{bmatrix}
         C_1 e^t \\
          C_2 e^{2t} \\
    \end{bmatrix} = \begin{bmatrix}
         y_1 \\
          y_2\\
    \end{bmatrix} \Longrightarrow \begin{bmatrix}
         y_1 \\
          y_2\\
    \end{bmatrix} = \begin{bmatrix}
         C_1 e^t + 3C_2 e^{2t}  \\
          C_2 e^{2t} \\
    \end{bmatrix}
\]
and we are done.
\end{longsolution}
This is a method for solving systems of linear differentiale equations, which is detailed as follows. Given a system \(\mathbf{y} ^{\prime} =A \mathbf{y} \) for \(A\) not diagonal but diagonalizable,
\begin{enumerate}
    \item Find a matrix \(P\) that diagonalizes \(A\).
    \item Let \(\mathbf{w} \coloneqq P ^{-1} \mathbf{y} \).
    \item Take \(\mathbf{y} ^{\prime} =P \mathbf{w} ^{\prime} \).
    \item Then \(P \mathbf{w} ^{\prime} =\mathbf{y} ^{\prime} =A \mathbf{y} = AP \mathbf{w}  \).
    \item Therefore \(\mathbf{w} ^{\prime} = P ^{-1} AP \mathbf{w} \).
    \item Since this system is diagonal, we can solve for \(\mathbf{w}\) using the earlier method.
    \item Plug in the solved \(\mathbf{w}  \) into the matrix equation \(\mathbf{y} =P \mathbf{w} \).
\end{enumerate}
\begin{exercise}
    Solve the system of first-order differential equations
    \[
        y_1^{\prime} =y_1 + 2y_2
    \]
    \[
        y_2^{\prime} = 2y_1 +y_2
    \]
\end{exercise}
\begin{solution}
    \[
        \begin{bmatrix}
             y_1^{\prime}  \\
              y_2^{\prime} \\
        \end{bmatrix} = \begin{bmatrix}
            1 &2   \\
             2&1   \\
        \end{bmatrix} \begin{bmatrix}
             y_1 \\
              y_2\\
        \end{bmatrix}
    \]
    Find eigenvalues.
    \[
        (\lambda -1)(\lambda -1)-4 = 0 \Longrightarrow \lambda =\pm\sqrt{4}+1
    \]
    \[
        \therefore \lambda \in \{ -1,3 \} 
    \]
    Find eigenvectors.
    \[
        \begin{bmatrix}
             -2&-2   \\
             -2&-2   \\
        \end{bmatrix} \begin{bmatrix}
             x \\
             y \\
        \end{bmatrix} = \begin{bmatrix}
             0 \\
             0 \\
        \end{bmatrix} \Longrightarrow \begin{bmatrix}
             -2x-2y \\
              -2x-2y\\
        \end{bmatrix} = \begin{bmatrix}
             0 \\
             0 \\
        \end{bmatrix} \Longrightarrow \begin{bmatrix}
             -2x \\
              -2x\\
        \end{bmatrix} = \begin{bmatrix}
             2y \\
             2y \\
        \end{bmatrix}
    \]
    Therefore the first eigenspace is spanned by \((-1,1)\) 
    \[
        \begin{bmatrix}
            2 &-2   \\
             -2&2   \\
        \end{bmatrix} \begin{bmatrix}
             x \\
             y \\
        \end{bmatrix} = \begin{bmatrix}
             2x - 2y \\
              -2x + 2y\\
        \end{bmatrix} = \begin{bmatrix}
             0 \\
             0 \\
        \end{bmatrix} \Longrightarrow \begin{bmatrix}
             2x \\
             -2x \\
        \end{bmatrix} = \begin{bmatrix}
             2y \\
             -2y \\
        \end{bmatrix}
    \]
    Therefore the second eigenspace is spanned by \((1,1)\). We then construct
    \[
        P\coloneqq \begin{bmatrix}
            -1 & 1  \\
            1 & 1  \\
        \end{bmatrix}
    \]
    Thus,
    \[
        P ^{-1} =\begin{bmatrix}
            -\frac{1}{2} &\frac{1}{2}   \\
             \frac{1}{2}&\frac{1}{2}   \\
        \end{bmatrix}
    \]
    We have
    \[
        \begin{bmatrix}
            -\frac{1}{2} &\frac{1}{2}   \\
             \frac{1}{2}&\frac{1}{2}   \\
        \end{bmatrix} \begin{bmatrix}
            1 &2   \\
             2&1   \\
        \end{bmatrix} \begin{bmatrix}
            -1 &1   \\
             1&1   \\
        \end{bmatrix} = \begin{bmatrix}
           -1  &0   \\
             0&3   \\
        \end{bmatrix}
    \]
    Define
    \[
        \mathbf{w} \coloneqq \begin{bmatrix}
            -\frac{1}{2} &\frac{1}{2}   \\
             \frac{1}{2}&\frac{1}{2}   \\
        \end{bmatrix} \begin{bmatrix}
             y_1 \\
             y_2 \\
        \end{bmatrix} = \begin{bmatrix}
             -\frac{1}{2}y_1 + \frac{1}{2}y_2 \\
              \frac{1}{2}y_1 + \frac{1}{2}y_2\\
        \end{bmatrix}
    \]
    Now give
    \[
        \mathbf{w} ^{\prime} \coloneqq \begin{bmatrix}
             -\frac{1}{2}y_1^{\prime} +\frac{1}{2}y_2^{\prime}  \\
              \frac{1}{2}y_1^{\prime} +\frac{1}{2}y_2^{\prime} \\
        \end{bmatrix}
    \]
    Thus
    \[
        \begin{bmatrix}
             w_1^{\prime}  \\
              w_2^{\prime} \\
        \end{bmatrix} = \begin{bmatrix}
            1 &0   \\
             0&3   \\
        \end{bmatrix} \begin{bmatrix}
             w_1 \\
              w_2\\
        \end{bmatrix} \Longrightarrow \begin{bmatrix}
             w_1 \\
              w_2\\
        \end{bmatrix} = \begin{bmatrix}
             Ce^t \\
              Ce^{3t} \\
        \end{bmatrix}
    \]
    Thus
    \[
        \begin{bmatrix}
             y_1 \\
             y_2 \\
        \end{bmatrix} = \begin{bmatrix}
            -\frac{1}{2} &\frac{1}{2}   \\
             \frac{1}{2}&\frac{1}{2}   \\
        \end{bmatrix} \begin{bmatrix}
             Ce^t \\
              Ce^{3t} \\
        \end{bmatrix} \Longrightarrow \begin{bmatrix}
             y_1 \\
              y_2\\
        \end{bmatrix} = \begin{bmatrix}
             -1Ce^t + Ce^{3t}  \\
              Ce^t + Ce^{3t} \\
        \end{bmatrix}
    \]
    and we are done.
\end{solution}
\begin{remark}
    A better way to remember it: diagonalize \(A\) via matrix \(P\) and get diagonal matrix \(D\). Then let some \(\mathbf{w} ^{\prime} = D \mathbf{w} \) and solve for the components of \(\mathbf{w} \). Then \(\mathbf{y} = P \mathbf{w} \).
\end{remark}