\section{Orthonormal Bases: Gram-Schmidt Process}
\begin{remark}
    When the inner product space being considered is \(\mathbb{R}^n\) or some subspace of \(\mathbb{R}^n\), assume the inner product being used is the Euclidean inner product.
\end{remark}
\noindent We say that the basis \(B=\{ (1,0,0),(0,1,0),(0,0,1) \} \) is the standard basis for \(\mathbb{R}^3\), since it's so convenient to use. Its convenience comes from the following properties:
\begin{enumerate}
    \item All vectors in \(B\) are \textbf{mutually orthogonal}, meaning
    \[
        (1,0,0)\cdot(0,1,0)=0
    \]
    \[
        (1,0,0)\cdot (0,0,1)=0
    \]
    \[
        (0,1,0)\cdot(0,0,1)=0
    \]
    \item All vectors in \(B\) are \textbf{unit} \textbf{vectors}.
\end{enumerate}
\begin{definition}[Orthogonal and Orthonormal Sets]
    A set \(S\) of vectors in an inner product space \(V\) is \textbf{orthogonal} when every pair of vectors in \(S\) is orthogonal. If every vector is also a unit vector, then \(S\) is orthonormal.
\end{definition}
If \(S\) is a basis for \(V\), we say it is an \textbf{orthogonal} \textbf{basis} or \textbf{orthonormal} \textbf{basis}, respectively. 
\begin{remark}
    There are multiple orthonormal bases for many vector spaces. Consider the following basis for \(\mathbb{R}^3\):
    \[
        B=\left\{ (\cos \theta ,\sin \theta ,0),(-\sin \theta ,\cos \theta ,0),(0,0,1) \right\}
    \]
\end{remark}
\begin{theorem}[Orthogonal Sets are Linearly Independent]\label{orthobasis}
    Let \(V\) be an inner product space with \(S=\left\{ \mathbf{v}_1,\ldots,\mathbf{v}_n \right\} \subset V \), where \(\mathbf{0}\notin S\). Then \(S\) is linearly independent.
\end{theorem}
\begin{proof}
    Since any linearly dependent set has some solution \(\left( c_1,\ldots,c_n \right)\neq (0,\ldots,0) \) for the equation 
    \[
        c_1 \mathbf{v}_1 +\cdots+c_n \mathbf{v}_n =\mathbf{0}
    \]
    we must show that this equation necessarily implies \(c_1 =\cdots=c_n=0\). Let \(\mathbf{v}_i\in S\) be an arbitrary member of the set. Without loss of generality, we can give
    \[
        \left\langle c_1 \mathbf{v}_1 +\cdots+c_i \mathbf{v}_i +\cdots+c_n \mathbf{v}_n \right\rangle = \left\langle \mathbf{0},\mathbf{v}_i \right\rangle
        \]
        By \ref{propyyy}, we have
        \[
        \Longrightarrow c_1 \left\langle \mathbf{v}_1,\mathbf{v}_i \right\rangle +\cdots+c_i \left\langle \mathbf{v}_i,\mathbf{v}_i \right\rangle +\cdots+c_n \left\langle \mathbf{v}_n,\mathbf{v}_i \right\rangle +\cdots=0
        \]
        By \ref{orthowogonality}, we have
        \[
            c_i \left\langle \mathbf{v}_i,\mathbf{v}_i \right\rangle =0
        \]
        By the definition of an inner product, we have 
        \[
            \left\langle \mathbf{v}_i,\mathbf{v}_i \right\rangle =0 \iff \mathbf{v}_i=\mathbf{0}
        \]
        However, by definition of \(S\), we have \(\mathbf{v}_i\neq 0\). Therefore,
        \[
            c_i \left\langle \mathbf{v}_i,\mathbf{v}_i \right\rangle =0 \Longrightarrow c_i=0
        \]
        Thus, all \(c_i=0\), and we have \(\left( c_1,\ldots,c_n \right)= (0,\ldots,0)\).
\end{proof}
\begin{corollary}\label{orthobasiscor}
    Let \(V\) be an inner product space with \(\dim (V)=n\). Any orthogonal set of \(n\) nonzero vectors is a basis for \(V\).
\end{corollary}
The above corrollary follows from the definition of a basis, and the proof is considered trivial. 
\begin{exercise}
    Using \ref{orthobasiscor}, show that \(S\) is a basis for \(\mathbb{R}^4\). 
    \[
        S=\left\{ (2,3,2,1),(1,0,0,1),(-1,0,2,1),(-1,2,-1,1) \right\} 
    \]
\end{exercise}
\begin{solution}
    \begin{align*}
        (2,3,2,1)\cdot(1,0,0,1)&=2+0+0-2=0\\
        (2,3,2,1)\cdot (-1,0,2,1)&=-2+0+4-2=0\\
        (2,3,2,1)\cdot(-1,2,-1,1)&=-2+6-2-2=0\\
        (1,0,0,1)\cdot(-1,0,2,1)&=-1+0+0+1=0\\
        (1,0,0,1)\cdot(-1,2,-1,1)&=-1+0+0+1=0\\
        (-1,0,2,1)\cdot(-1,2,-1,1)&=1+0-2+1=0
    \end{align*}
    By \ref{orthobasiscor}, \(S\) is a basis for \(\mathbb{R}^4\).
\end{solution}
\begin{theorem}[Coordinates Relative to an Orthonormal Basis]
    Let \(B=\left\{ \mathbf{v}_1,\ldots,\mathbf{v}_n \right\} \) be an orthonormal basis for an inner product space \(V\). The coordinate representation of a vector \(\mathbf{w}\) relative to \(B\) is given as
    \[
        \mathbf{w}=\left\langle \mathbf{w},\mathbf{v}_1 \right\rangle \mathbf{v}_{1}+\cdots+\left\langle \mathbf{w},\mathbf{v}_n \right\rangle \mathbf{v}_n
    \]
\end{theorem}
\begin{proof}
    Let \(B\) be an orthonormal basis for an inner product space \(V\) with \(\mathbf{w}\in V\). It follows from the definition of a basis that there exists scalars \(c_1,\ldots,c_n\in\mathbb{F}\) such that
    \[
        \mathbf{w}=c_1 \mathbf{v}_1 +\cdots+c_n \mathbf{v}_n
    \]
    It follows that 
    \begin{align*}
        \left\langle \mathbf{w},\mathbf{v}_i \right\rangle &=\left\langle \left( c_1 \mathbf{v}_1 +\cdots+c_n \mathbf{v}_n \right),\mathbf{v}_i  \right\rangle\\
        &= c_1 \left\langle \mathbf{v}_1,\mathbf{v}_i \right\rangle +\cdots c_i\left\langle \mathbf{v}_i,\mathbf{v}_i \right\rangle +\cdots c_n\left\langle \mathbf{v}_n,\mathbf{v}_i \right\rangle\\
        &= c_i \left\langle \mathbf{v}_i,\mathbf{v}_i \right\rangle 
    \end{align*}
    Since \(B\) is orthonormal with \(\mathbf{v}_i\in B\), we have 
    \[
        \left\langle \mathbf{w}=\mathbf{v}_i \right\rangle =c_i \left\langle \mathbf{v}_i,\mathbf{v}_i \right\rangle =c_i \left\lVert \mathbf{v}_i \right\rVert =c_i
    \]
    \[
        \therefore \left\langle \mathbf{w},\mathbf{v}_i \right\rangle =c_i
    \]
\end{proof}
The \textbf{Gram-Schmidt orthonormalization process} is a systematic method for constructing orthonormal bases from some arbitrary basis.
\begin{theorem}[Gram-Schmidt Orthonormalization Process]\label{gram}
    Let \(V\) be an inner product space with a basis \(B=\{ \mathbf{v}_1,\ldots,\mathbf{v}_n \} \). Define the set \(B^{\prime} =\{ \mathbf{w}_1,\ldots,\mathbf{w}_n \} \), where
    \[
        \begin{array}{ccc}
            \mathbf{w}_1&=&\mathbf{v}_1\\
            \mathbf{w}_2&=&\mathbf{v}_2 -\frac{\left\langle \mathbf{v}_2,\mathbf{w}_1 \right\rangle }{\left\langle \mathbf{w}_1,\mathbf{w}_1 \right\rangle }\mathbf{w}_1\\
            \mathbf{w}_3&=&\mathbf{v}_3 -\frac{\left\langle \mathbf{v}_3,\mathbf{w}_1 \right\rangle }{\left\langle \mathbf{w}_1,\mathbf{w}_1 \right\rangle }\mathbf{w}_1 -\frac{\left\langle \mathbf{v}_3,\mathbf{w}_2 \right\rangle }{\left\langle \mathbf{w}_2,\mathbf{w}_2 \right\rangle }\mathbf{w}_2\\
            &\vdots&\\
            \mathbf{w}_n&=&\mathbf{v}_n -\frac{\left\langle \mathbf{v}_n ,\mathbf{w}_1\right\rangle }{\left\langle \mathbf{w}_1,\mathbf{w}_1 \right\rangle }\mathbf{w}_1 - \frac{\left\langle \mathbf{v}_n,\mathbf{w}_2 \right\rangle }{\left\langle \mathbf{w}_2,\mathbf{w}_2 \right\rangle }\mathbf{w}_2 -\cdots-\frac{\left\langle \mathbf{v}_n,\mathbf{w}_{n-1}  \right\rangle }{\left\langle \mathbf{w}_{n-1},\mathbf{w}_{n-1}   \right\rangle }\mathbf{w}_{n-1} 
        \end{array}
    \]
    Then \(B^{\prime} \) is an orthogonal basis for \(V\). To normalize \(B^{\prime} \), let \(B^{\prime\prime} \) be the set \(B^{\prime\prime} =\left\{ \mathbf{u}_1,\ldots,\mathbf{u}_n \right\} \), where 
    \[
        \mathbf{u}_i =\frac{\mathbf{w}_i}{\left\lVert \mathbf{w}_i \right\rVert }
    \]
    Then \(B^{\prime\prime} \) is an orthonormal basis for \(V\).
\end{theorem}
``Rather than give a proof of this theorem, it is more instructive to discuss a special case for which you can use a geometric model.'' -the textbook lmfao
\begin{longproof}
    I honestly don't know how to derive the formula, but we can use proof by induction to verify that \(B^{\prime} \) is an orthogonal basis.\\
    Let \(B=\left\{ \mathbf{v}_1,\ldots,\mathbf{v}_n \right\} \) be a basis for an inner product space \(V\). Let \(\mathbf{w}_1 \coloneqq \mathbf{v}_1\). Let
    \[
        \mathbf{w}\coloneqq \mathbf{v}_2 -\frac{\left\langle \mathbf{v}_2,\mathbf{w}_1 \right\rangle }{\left\langle \mathbf{w}_1,\mathbf{w}_1 \right\rangle }\mathbf{w}_1
    \]
    It follows that
    \begin{align*}
        \left\langle \mathbf{w}_1,\mathbf{w}_2 \right\rangle &= \left\langle \mathbf{w}_1,\mathbf{v}_2 -\frac{\left\langle \mathbf{v}_2,\mathbf{w}_1 \right\rangle }{\left\langle\mathbf{w}_1,\mathbf{w}_1\right\rangle}\mathbf{w}_1 \right\rangle \\
        &=\left\langle \mathbf{w}_1,\mathbf{v}_2 \right\rangle -\left\langle \mathbf{w}_1,\frac{\left\langle \mathbf{v}_2,\mathbf{w}_1 \right\rangle }{\left\langle\mathbf{w}_1,\mathbf{w}_1\right\rangle}\mathbf{w}_1 \right\rangle \\
        &= \left\langle \mathbf{w}_1,\mathbf{v}_2 \right\rangle -\frac{\left\langle \mathbf{v}_2,\mathbf{w}_1 \right\rangle }{\left\langle\mathbf{w}_1,\mathbf{w}_1\right\rangle}\left\langle \mathbf{w}_1,\mathbf{w}_1 \right\rangle \\
        &= \left\langle \mathbf{w}_1,\mathbf{v}_2 \right\rangle -\frac{\left\langle \mathbf{v}_2,\mathbf{w}_1 \right\rangle}{\left\lVert \mathbf{w}_1 \right\rVert }\left\lVert \mathbf{w}_1 \right\rVert \\
        &=\left\langle \mathbf{w}_1,\mathbf{v}_2 \right\rangle - \left\langle \mathbf{w}_1,\mathbf{v}_2 \right\rangle\\
        &=0
    \end{align*}
    Therefore, \(\mathbf{w}_1\) is orthogonal to \(\mathbf{w}_2\).
    Now suppose we have defined \(k<n\) orthogonal vectors. We must show that this implies the \(k+1\)th vector will be orthogonal to all other vectors. Using the above theorem, let 
    \[
        \mathbf{w}_{k+1}  \coloneqq \mathbf{v}_{k+1} - \frac{\left\langle \mathbf{v}_{k+1},\mathbf{w}_1 \right\rangle }{\left\langle \mathbf{w}_1,\mathbf{w}_1 \right\rangle }\mathbf{w}_1 - \frac{\left\langle \mathbf{v}_{k+1},\mathbf{w}_2 \right\rangle }{\left\langle \mathbf{w}_2,\mathbf{w}_2 \right\rangle }\mathbf{w}_2 -\cdots- \frac{\left\langle \mathbf{v}_{k+1},\mathbf{w}_{k}  \right\rangle }{\left\langle \mathbf{w}_{k},\mathbf{w}_{k}   \right\rangle }\mathbf{w}_{k} 
    \]
    Now consider \(\mathbf{w}_i\), where \(1\leq i\leq k\).
    \begin{align*}
        \left\langle \mathbf{w}_i,\mathbf{w}_{k+1}  \right\rangle &=\left\langle \mathbf{w}_i,\mathbf{v}_{k+1} - \frac{\left\langle \mathbf{v}_{k+1},\mathbf{w}_1 \right\rangle }{\left\langle \mathbf{w}_1,\mathbf{w}_1 \right\rangle }\mathbf{w}_1 - \frac{\left\langle \mathbf{v}_{k+1},\mathbf{w}_2 \right\rangle }{\left\langle \mathbf{w}_2,\mathbf{w}_2 \right\rangle }\mathbf{w}_2 -\cdots- \frac{\left\langle \mathbf{v}_{k+1},\mathbf{w}_{k}  \right\rangle }{\left\langle \mathbf{w}_{k},\mathbf{w}_{k}   \right\rangle }\mathbf{w}_{k} \right\rangle \\
        &=\left\langle \mathbf{w}_i, \mathbf{v}_{k+1}-\frac{\left\langle\mathbf{v}_{k+1},\mathbf{w}_i\right\rangle }{\left\langle\mathbf{w}_i,\mathbf{w}_i\right\rangle}\mathbf{w}_i \right\rangle \,\because \mathbf{w}_i \text{ is orthogonal to all }\mathbf{w}_j,i\neq j\\
        &=\left\langle \mathbf{w}_i,\mathbf{v}_{k+1}  \right\rangle - \left\langle \mathbf{w}_i,\frac{\left\langle\mathbf{v}_{k+1},\mathbf{w}_i\right\rangle }{\left\langle\mathbf{w}_i,\mathbf{w}_i\right\rangle}\mathbf{w}_i \right\rangle \\
        &=\left\langle \mathbf{w}_i,\mathbf{v}_{k+1}  \right\rangle - \frac{\left\langle\mathbf{v}_{k+1},\mathbf{w}_i\right\rangle }{\left\langle\mathbf{w}_i,\mathbf{w}_i\right\rangle} \left\langle \mathbf{w}_i,\mathbf{w}_i \right\rangle \\
        &=\left\langle \mathbf{w}_i,\mathbf{v}_{k+1}  \right\rangle - \frac{\left\langle\mathbf{v}_{k+1},\mathbf{w}_i\right\rangle }{\left\lVert \mathbf{w}_i \right\rVert }\left\lVert \mathbf{w}_i \right\rVert \\
        &=\left\langle \mathbf{w}_i,\mathbf{v}_{k+1}  \right\rangle - \left\langle \mathbf{v}_{k+1},\mathbf{w}_i  \right\rangle \\
        &=\left\langle \mathbf{w}_i,\mathbf{v}_{k+1}  \right\rangle - \left\langle \mathbf{w}_i,\mathbf{v}_{k+1}  \right\rangle\\
        &=0
    \end{align*}
    \[
        \therefore \mathbf{w}_i \perp \mathbf{w}_{k+1}\,\forall 1\leq i\leq k 
    \]
    When constructing a basis, we simply stop the process for \(k=n\). Thus, by induction, this step of \ref{gram} produces an orthogonal basis. Simply normalizing each vector will preserve orthogonality, but will produce an orthonormal basis. Thus, the proof is complete.
\end{longproof}
\begin{remark}
    An orthonormal set derived using \ref{gram} will be different depending on the ordering of the vectors in the basis.
\end{remark}
\begin{exercise}
    Use Gram-Schmidt orthonormalization on the basis \(B=\left\{ 1,x,x^2 \right\} \) on the vector space \(P_2\), with the inner product
    \[
        \langle p,q \rangle = \int_{-1}^1 p(x)q(x)\,dx
    \]
\end{exercise}
\begin{longsolution}
    Let \(B^{\prime} =\left\langle \mathbf{w}_1,\mathbf{w}_2,\mathbf{w}_3 \right\rangle \) be the set obtained after applying the first step of Gram-Schmidt. Let
    \[
        \mathbf{v}_1 = 1\quad \mathbf{v}_2 = x\quad \mathbf{v}_3 = x^2
    \]
    We have
    \[
        \mathbf{w}_1 = \mathbf{v}_1 = 1
    \]
    Thus, we also have
    \begin{align*}
        \mathbf{w}_2 &= \mathbf{v}_2 -\frac{\left\langle \mathbf{v}_2,\mathbf{w}_1 \right\rangle }{\left\langle \mathbf{w}_1,\mathbf{w}_1 \right\rangle }\mathbf{w}_1\\
        &= x - \frac{\int_{-1}^1 x\,dx}{\int_{-1}^1 \,dx}1\\
        &= x
    \end{align*}
    Finally, we have
    \begin{align*}
        \mathbf{w}_{3}&= \mathbf{v}_3 - \frac{\left\langle \mathbf{v}_3,\mathbf{w}_1 \right\rangle }{\left\langle\mathbf{w}_1,\mathbf{w}_1 \right\rangle}\mathbf{w}_1 - \frac{\left\langle \mathbf{v}_3,\mathbf{w}_2 \right\rangle }{\left\langle\mathbf{w}_{2},\mathbf{w}_2 \right\rangle}\mathbf{w}_2\\
        &=x^2 - \frac{\int_{-1}^1 x^2\,dx}{\int_{-1}^1 \,dx}1 - \frac{\int_{-1}^1 x^3 }{\int_{-1}^1 x^2}\,dx\\
        &=x^2 - \frac{1}{3}
    \end{align*}
    Let \(B^{\prime\prime} =\left\{ \mathbf{u}_1,\mathbf{u}_2,\mathbf{u}_3 \right\} \), where 
    \[
        \mathbf{u}_1 = \frac{\mathbf{w}_1}{\left\lVert \mathbf{w}_1 \right\rVert }\quad\mathbf{u}_2 = \frac{\mathbf{w}_2}{\left\lVert \mathbf{w}_2 \right\rVert }\quad\mathbf{u}_3 = \frac{\mathbf{w}_3}{\left\lVert \mathbf{w}_3 \right\rVert }
    \]
    We have
    \begin{align*}
        \mathbf{u}_1 &= \frac{\mathbf{w}_1}{\left\lVert \mathbf{w}_1 \right\rVert }=\frac{1}{\sqrt{2} }\\
        \mathbf{u}_2 &=\frac{\mathbf{w}_2}{\left\lVert \mathbf{w}_2 \right\rVert }=\frac{\sqrt{3} }{\sqrt{2} }x\\
        \mathbf{u}_3 &= \frac{\mathbf{w}_3}{\left\lVert \mathbf{w}_3 \right\rVert }=\frac{\sqrt{5} }{2\sqrt{2} }\left( 3x^2 -1 \right) 
    \end{align*}
\end{longsolution}
\begin{remark}
    The polynomials \(\mathbf{u}_1,\mathbf{u}_2,\mathbf{u}_3\) in the previous example are the first three \textbf{normalized} \textbf{Legendre} \textbf{polynomials}.
\end{remark}