\chapter{Eigenvalues and Eigenvectors}
\section{Eigenvalues and Eigenvectors}
Sometimes, for a linear transformation \(T:\mathbb{R} ^n \to \mathbb{R} ^n\) given by \(A\in M_{n,n}\), there exists some \(\lambda\in \mathbb{R}  \) and nonzero \(\mathbf{v} \in\mathbb{R} ^n\) such that 
\[
    A \mathbf{x} =\lambda  \mathbf{x} 
\]
These values are extremely important, and are known as eigenvalues and eigenvectors.
\begin{definition}[Eigenvalues and Eigenvectors]
    Let \(A\in M_{n,n}\). If there exists a nonzero vector \(\mathbf{x} \) and scalar \(\lambda \) such that \(A \mathbf{x} =\lambda  \mathbf{x} \), then we call \(\lambda \) an \textbf{eigenvalue} of \(A\), and \(\mathbf{x} \) is the \textbf{eigenvector} corrsponding to \(\lambda \).
\end{definition}	
\begin{theorem}[Eigenspace]
    Let \(A\in M_{n,n}\) have eigenvalue \(\lambda \). The set
    \[
        E\coloneqq \left\{ \mathbf{x} \mid A \mathbf{x} =\lambda \mathbf{x} \land \mathbf{x} \neq \mathbf{0}  \right\} \cup \{ \mathbf{0}  \}
    \]
    is a subspace of \(\mathbb{R} ^n\), called the \textbf{eigenspace} of \(\lambda \).
\end{theorem}
\begin{proof}
    Let \(\mathbf{u} ,\mathbf{v} \in E\). Then \(A \mathbf{u} =\lambda  \mathbf{u} \) and \(A \mathbf{v} =\lambda \mathbf{v} \). It follows that 
    \begin{align*}
        A(\mathbf{u} +\mathbf{v} )&=A \mathbf{u} +A \mathbf{v} \\
        &=\lambda \mathbf{u} +\lambda \mathbf{v} \\
        &=\lambda (\mathbf{u} +\mathbf{v} )
    \end{align*}
    Therefore, \(A(\mathbf{u} +\mathbf{v} )=\lambda (\mathbf{u} +\mathbf{v} )\), and thus \(\mathbf{u} +\mathbf{v} \in E\). Similarly, let \(a\in\mathbb{R} \) and observe that
    \begin{align*}
        A(a \mathbf{v} )&=a(A \mathbf{v} )\\
        &=a(\lambda \mathbf{v} )\\
        &=\lambda (a \mathbf{v} )
    \end{align*}
    Therefore, \(A(a \mathbf{v} )=\lambda (a \mathbf{v} )\), and thus \(a \mathbf{v} \in E\). Finally, \(\mathbf{0}\in E \) by definition. Since \(E\) is closed under vector addition, scalar multiplication, and is nonempty, \(E\) forms a subspace of \(\mathbb{R} ^n\).
\end{proof}
\begin{theorem}[Eigenvalues and Eigenvectors of a Matrix]
    Let \(A\in M_{n,n}\). An eigenvalue of \(A\) is a scalar \(\lambda \) such that \(\det \left( \lambda I - A \right)=0 \), and the eigenvectors of \(A\) corresponding to \(\lambda \) are the nonzero solutions of \(\left( \lambda I-A \right)\mathbf{x} =\mathbf{0}  \).
\end{theorem}
\begin{proof}
    Let \(A \mathbf{x} =\lambda \mathbf{x} \). We have
    \[
        A \mathbf{x} = (\lambda I)\mathbf{x}
    \]
    \[
        \Longrightarrow A \mathbf{x} -(\lambda I)\mathbf{x} = \mathbf{0} 
    \]
    \[
        \Longrightarrow (A-\lambda I)\mathbf{x} =\mathbf{0} 
    \]
    If \(A-\lambda I\) is invertible, there will only exist the solution \(\mathbf{x} =\mathbf{0} \). Thus, we require \(\det (A-\lambda I)=0\).
\end{proof}
\begin{remark}
    The equation \(\det (\lambda I-A)=0\) is called the characteristic equation of \(A\), and the corresponding polynomial \(\lambda ^n + c_{n-1}\lambda^{n-1}+\cdots+c_1 \lambda  + c_0=\det (\lambda I-A ) \) is called the caracteristic polynomial of \(A\).
\end{remark}
The process for finding eigenvalues and eigenvectors for some \(A\in M_{n,n}\) is summarized as
\begin{enumerate}
    \item Form the characteristic equation \(\det (\lambda I-A)=0\), a polynomial of degree \(n\).
    \item Find the \(n\) roots guarenteed by the Fundamental Theorem of Algebra (this class only focuses on real roots but complex roots are still eigenvalues)
    \item For each \(\lambda _i\), solve the homogeneous system \((\lambda_i I - A)\mathbf{x} =\mathbf{0} \) via rrefing an \(n\times n\) matrix to determine all eigenvectors of \(\lambda_i\).
\end{enumerate}
\begin{theorem}[Eigenvalues of Triangular Matrices]
    If \(A\in M_{n,n}\) is triangular, then the eigenvalues of \(A\) are the entries on its diagonal.
\end{theorem}
\section{Diagonalization}