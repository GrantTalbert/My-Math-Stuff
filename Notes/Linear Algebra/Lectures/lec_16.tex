\section{Symmetric Matrices and Orthogonal Diagonalization}
\begin{definition}[Symmetric Matrices]
    A symmetric matrix \(A\) is any matrix such that \(A=A^T\).
\end{definition}
\begin{theorem}[Real Spectral Theorem]
    Let \(A\in M_{n\times n}\) be symmetric. Then
    \begin{itemize}
        \item \(A\) is diagonalizable.
        \item All eigenvalues of \(A\) are real.
        \item If \(\lambda \) is an eigenvalue of \(A\) with multiplicity \(k\), then the eigenspace of \(\lambda \) has dimension \(k\).
    \end{itemize}
\end{theorem}
\begin{definition}[Orthogonal Matrices]
    Let \(A\in \operatorname{GL}_n(\mathbb{R} ) \). We say that \(A\) is orthogonal iff \(A=A^{-1} \).
\end{definition}
\begin{theorem}
    A square matrix is orthogonal if and only if its columns form an orthonormal set.
\end{theorem}
\begin{theorem}[Property of Symmetric Matrices]
    Let \(A\) be symmetric with distinct eigenvalues \(\lambda _1,\lambda _2\). If \(\mathbf{x} _1\) and \(\mathbf{x} _2\) are the eigenvectors corresponding to \(\lambda _1\) and \(\lambda _2\), then \(\mathbf{x} _1\) and \(\mathbf{x} _2\) are orthogonal.
\end{theorem}
\begin{proof}
    Let \(\lambda _1 \mathbf{x} _1 = A \mathbf{x} _1\) and \(\lambda _2 \mathbf{x} _2 = A \mathbf{x} _2\) with \(\lambda _1 \neq \lambda _2\). We have
    \begin{align*}
        \lambda_1 (\mathbf{x} _1 \cdot \mathbf{x} _2)&= (\lambda _1 \mathbf{x} _1)\cdot \mathbf{x} _2\\
        &=A \mathbf{x} _1 \cdot \mathbf{x} _2\\
        &\cong (A \mathbf{x} _1)^T \mathbf{x} _2\\
        &= \mathbf{x} _1^T A^T \mathbf{x} _2\\
        &=\mathbf{x} _1^T A \mathbf{x} _2\\
        &=\mathbf{x} _1^T \lambda \mathbf{x} _2\\
        &=\lambda _2 \mathbf{x} _1^T \mathbf{x} _2\\
        &\cong \lambda _2(\mathbf{x} _1 \cdot \mathbf{x} _2)
    \end{align*}
    Therefore, \(\lambda_1 (\mathbf{x} _1\cdot \mathbf{x} _2)=\lambda _2 (\mathbf{x} _1\cdot \mathbf{x} _2)\). Letting \(\mathbf{x} _1\cdot \mathbf{x}_2 \neq 0\) implies that \(\lambda _1 =\lambda _2 \Longrightarrow \Longleftarrow\), so \(\mathbf{x} _1 \cdot \mathbf{x} _2 = 0\).
\end{proof}
\begin{definition}[Orthogonal Diagonalization]
    A matrix \(A\) is orthogonally diagonalizable if there exists an orthogonal matrix \(P\) with \(P ^{-1} AP=D\) being diagonal.
\end{definition}	
\begin{theorem}[Fundamental Theorem of Symmetric Matrices]
    Let \(A\) be a square matrix. Then \(A\) is orthogonally diagonalizable if and only if \(A\) is symmetric.
\end{theorem}
Let's take a break from the rapid fire definitions and theorems (not proofs for some reason) to go over a method for orthogonally diagonalizing a matrix. Let \(A\in M_{n\times n}\) be symmetric.
\begin{enumerate}
    \item Find all eigenvalues of \(A\) and determine their multiplicity.
    \item Find each eigenvector corresponding to each eigenvalue. Recall that for eigenvalues of multiplicity \(\neq 1\), there will be more than 1 eigenvector. Normalize each eigenvector.
    \item If the set is not orthonormal, apply the Gram-Schmidt orthonormalization process.
    \item This will product a set of \(n\) orthonormal eigenvectors. Use these to form the columns of \(P\), and the resulting matrix of \(P ^{-1} AP=D\) will have the eigenvalues on the diagonal, with an order corresponding to the order of the eigenvectors.
\end{enumerate}