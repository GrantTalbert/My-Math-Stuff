\section{The Kernel and Range of a Linear Transformation}
\begin{definition}[Kernel]
    Let \(T:V\to W\) be a linear transformation. The set of all \(\mathbf{v}\in V\) such that \(T(\mathbf{v})=\mathbf{0}\) is the kernel of \(T\).
    \[
        \ker (T)=\left\{ \mathbf{v}\in V\mid T(\mathbf{v})=\mathbf{0} \right\} 
    \]
\end{definition}
\begin{exercise}
    Find the kernel of the transformation \(T:\mathbb{R}^3 \to \mathbb{R}^3\) given by \(T(x,y,z)=(0,0,0)\).
\end{exercise}
\begin{solution}
    Since any \(v\in\mathbb{R}^3\) has \(T(\mathbf{v})=\mathbf{0}\), we have 
    \[
        \ker (T)=\mathbb{R}^3
    \]
\end{solution}
\begin{exercise}
    Find the kernel of the transformation \(T:\mathbb{R}^2 \to \mathbb{R}^2\) given by \(T(x,y)=T(x-y,y-x)\).
\end{exercise}
\begin{solution}
    \[
        x-y=0\land y-x=0\iff x=y
    \]
    \[
        \therefore \ker (t)=\left\{ (x,y)\in\mathbb{R} \mid x=y \right\} 
    \]
\end{solution}
\begin{center}
    \begin{figure}[ht]
        \scalebox{.75}{\incfig{Transform}}
        \centering
        \caption{The Kernel of a Linear Map \(T\)}
    \end{figure}
\end{center}
\begin{theorem}
    Let \(T:V\to W\) be a linear transformation. Then \(\ker (T)\) is a subspace of the domain \(V\).
\end{theorem}
\begin{proof}
    Let \(\mathbf{u},\mathbf{v}\in \ker (T) \subset V\). Let \(c\in\mathbb{F}\). By \ref{Propowowo}, 
    \[
        T(\mathbf{0})=\mathbf{0}\Longrightarrow \mathbf{0}\in\ker (T)\,\forall T
    \]
    \[
        \therefore \ker (T)\neq \varnothing 
    \]
    \begin{align*}
        \mathbf{u},\mathbf{v}\in\ker (T)&\Longrightarrow T(\mathbf{u})=T(\mathbf{v})=\mathbf{0}\\
        &\Longrightarrow T(\mathbf{u})+T(\mathbf{v})=\mathbf{0}\\
        &\Longrightarrow T(\mathbf{u}+\mathbf{v})=\mathbf{0}\\
        &\Longrightarrow \mathbf{u}+\mathbf{v}\in\ker (T)
    \end{align*}
    \begin{align*}
        T(c \mathbf{v})&=cT(\mathbf{v})\\
        &=c \mathbf{0}\\
        &=\mathbf{0}\\
        &\Longrightarrow c\mathbf{v}\in\ker (T)
    \end{align*}
    \[
        \therefore \ker (T)\text{ is a subspace of }V 
    \]
\end{proof}
\begin{corollary}\label{cor:6.7}
    Let \(A\) be an \(m\times n\) matrix and \(T:\mathbb{R}^n \to \mathbb{R}^m\) be the linear transformation \(T(\mathbf{x})=A \mathbf{x}\). Then the kernel of \(T\) is equal to the solution space of \(A \mathbf{x}=\mathbf{0}\).
\end{corollary}
\begin{exercise}
    Find a basis for the kernel of the linear transformation \(T:\mathbb{R}^3 \to \mathbb{R}^2\) given by 
    \[
        T(\mathbf{x})= \begin{bmatrix}[r]
            -1 &-2  &1   \\
             0&  2&   1\\
        \end{bmatrix} \begin{bmatrix}[r]
             x \\
             y \\
             z \\
        \end{bmatrix}
    \]
\end{exercise}
\begin{solution}
    Let the above matrix be called \(A\). By \ref{cor:6.7}, we know that
    \[
        \ker (T)=\left\{ \begin{bmatrix}[r]
             x \\
             y \\
             z \\
        \end{bmatrix}\in\mathbb{R}^3 \mid \begin{bmatrix}[r]
            -1 &-2  &1   \\
             0&  2&   1\\
        \end{bmatrix} \begin{bmatrix}[r]
            x \\
            y \\
            z \\
       \end{bmatrix}=\begin{bmatrix}[r]
         0 \\
         0 \\
       \end{bmatrix} \right\} 
    \]
    We can find a basis for this set. First rref \(A\):
    \[
        \begin{bmatrix}[r]
            -1 &-2  &1   \\
             0&  2&   1\\
        \end{bmatrix}\xlongrightarrow{rref} \begin{bmatrix}[r]
            1 &0  &2   \\
             0&1  &0.5   \\
        \end{bmatrix}
    \]
    Notice that the only column that isn't a pivot column is the third column, so our free variable will be \(z\). Set 
    \[
        z=t
    \]
    and solve. 
    \[
        x+2t = 0 \Longrightarrow  x=-2t
    \]
    \[
        y+\frac{1}{2}t=0 \Longrightarrow y=-\frac{1}{2}t
    \]
    Therefore, we have the general form of a solution as
    \[
        \mathbf{x}=\begin{bmatrix}[r]
             -2t \\
             -\frac{1}{2}t \\
              t\\
        \end{bmatrix}
    \]
    Since we only have one free variable, this column vector will be the basis for our set, when we set \(t=1\):
    \[
        \ker (T)=N(A)= \text{span} \left\{ \begin{bmatrix}[r]
             -2 \\
             -\frac{1}{2} \\
              1\\
        \end{bmatrix} \right\} 
    \]
\end{solution}
\begin{definition}[Range]
    The range of a linear transformation \(T:V\to W\) is the set of all \(\mathbf{w}\in W\) that have a preimage under \(T\) in \(V\). That is,
    \[
        \text{range}(T)=\left\{ T(\mathbf{v}) \mid \mathbf{v}\in V \right\} 
    \]
\end{definition}
\begin{theorem}
    The range of a linear transformation \(T:V\to W\) is a subspace of \(W\).
\end{theorem}
\begin{proof}
    By \ref{Propowowo}, all linear maps must give \(\mathbf{0}\mapsto \mathbf{0}\), and thus \(0\in\operatorname{range}(T)\neq \varnothing  \). Let \(\mathbf{u},\mathbf{v}\in V\) and \(\mathbf{x},\mathbf{y}\in W\) such that 
    \[
        T(\mathbf{u})=\mathbf{x}\qquad T(\mathbf{v})=\mathbf{y}
    \]
    We have
    \[
        \mathbf{x}+\mathbf{y}=T(\mathbf{u})+T(\mathbf{v})=T(\mathbf{u}+\mathbf{v})
    \]
    Since \(V\) is a vector space, \(\mathbf{u}+\mathbf{v}\in V\), and thus there exists some \(\mathbf{u}+\mathbf{v}\in V\) such that \(\mathbf{x}+\mathbf{y}=T(\mathbf{u}+\mathbf{v})\). Finally, let \(c\in\mathbb{F}\). For any \(c \mathbf{x}\), we have 
    \[
        c \mathbf{x}=c T(\mathbf{u})=T(c \mathbf{u})
    \]
    Since \(V\) is a vector space, \(c \mathbf{u}\in V\). Thus, \(c \mathbf{u}\in \operatorname{range}(T) \). Therefore, \(\operatorname{range}(T) \) is a subspace of \(W\).
\end{proof}
\begin{corollary}
    Let \(A\in M_{m,n}\) give the linear transformation \(T:\mathbb{R}^n \to \mathbb{R}^m\). Then the column space of \(A\) is equal to the range of \(T\).
\end{corollary}
\begin{center}
\begin{figure}[ht]
    \incfig{Fig}
    \centering
    \caption{The Range of a Linear Map \(T\)}
\end{figure}
\end{center}