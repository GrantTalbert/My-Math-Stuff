\begin{remark}
    The identity map \(\text{id}_{\mathbb{R}^n} \) is given by the identity matrix \(I_n\).
\end{remark}
\begin{theorem}[Existence of an Inverse Linear Map]
    Let \(T:\mathbb{R}^{n}\to \mathbb{R}^n \) be a linear transformation with standard matrix \(A\). The following conditions are equivalent. 
    \begin{enumerate}
        \item \(T\) is invertible. 
        \item \(T\) is an isomorphism.
        \item \(A\) is invertible.
    \end{enumerate}
\end{theorem}
\begin{exercise}
    Determine if \(T(x,y)=(2x,0)\) is invertible.
\end{exercise}
\begin{solution}
    \(T(1,0)=(2,0)\) and \(T(0,1)=(0,0)\), so 
    \[
        A=\begin{bmatrix}[r]
            2 & 0  \\
             0&0   \\
        \end{bmatrix}
    \]
    no thats not invertible lol
\end{solution}
\begin{exercise}
    Determine if \(T(x,y)=(x+y,x-y)\) is invertible. If it is, find its inverse.
\end{exercise}
\begin{solution}
    We have \(T(1,0)=(1,1)\) and \(T(0,1)=(1,-1)\), so if \(T:\mathbf{x}\mapsto A \mathbf{x}\), we have
    \[
        A=\begin{bmatrix}[r]
            1 &1   \\
             1&-1   \\
        \end{bmatrix}
    \]
    Notice \(\det (A)=-2\neq 0\), so \(T\) is invertible. We give 
    \[
        A^{-1} = \frac{1}{-2} \begin{bmatrix}[r]
            -1 &-1   \\
             -1&1   \\
        \end{bmatrix}
    \]
    It follows that 
    \[
        T^{-1} (\mathbf{x})= A^{-1} \mathbf{x}
    \]
\end{solution}
\begin{theorem}[Transformation Matrix for Nonstandard Bases]\label{thismakessense}
    Let \(V\) and \(W\) be finite dimensional vector spaces with arbitrary bases \(\mathcal{B} \) and \(\mathcal{B} ^{\prime} \), respectively. Let 
    \[
        \mathcal{B} = \left\{ \mathbf{v}_1,\mathbf{v}_2,\ldots,\mathbf{v}_n \right\} 
    \]
    Let \(T:V\to W\) be a linear map with 
    \[
        [T(\mathbf{v}_1)]_{\mathcal{B} ^{\prime} }=\begin{bmatrix}[r]
             a_{11}  \\
              a_{21} \\
              \vdots\\
              a_{m1} \\
        \end{bmatrix}\quad [T(\mathbf{v}_2)]_{\mathcal{B} ^{\prime} }=\begin{bmatrix}[r]
            a_{12}  \\
             a_{22} \\
             \vdots\\
             a_{m2} \\
       \end{bmatrix}\quad \cdots \quad [T(\mathbf{v_n})]_{\mathcal{B} ^{\prime} }=\begin{bmatrix}[r]
        a_{1n}  \\
         a_{2n} \\
         \vdots\\
         a_{mn} \\
   \end{bmatrix}
    \]
    Define the matrix
    \[
        A=\begin{bmatrix}[r]
            a_{11}  &a_{12}   &\cdots  &a_{1n}    \\
             a_{21} &a_{22}   &\cdots  &a_{2n}    \\
             \vdots&\vdots  &\ddots  &\vdots   \\
             a_{m1} &a_{m2}   &\cdots &a_{mn}    \\
        \end{bmatrix}
    \]
    It follows that for all \(\mathbf{v}\in V\),
    \[
        [T(\mathbf{v})]_{\mathcal{B} ^{\prime} }=A[\mathbf{v}]_\mathcal{B} 
    \]
\end{theorem}
This theorem is fairly confusing. This is basically a generalization of the matrix representation of a linear map to an arbitrary basis. If one defines a map \(T\) and identifies \(T(\mathbf{e}_i)\) relative to whatever basis for \(W\) is being considered for all basis elements \(\mathbf{e}_i\), then the matrix \(A\) gives a map from the basis \(\mathcal{B} \) to \(\mathcal{B} ^{\prime} \) at the same time as \(V\to W\). The last line of the theorem sums this up and is likely the most important line.
\begin{exercise}
    Let \(T:\mathbb{R}^2 \to \mathbb{R}^3\) be a map with \(T(x,y)=(x+y,x,y)\). Find the matrix \(A\in M_{m,n}\) where \(T(\mathbf{x})=A \mathbf{x}\) relative to the bases \(\mathcal{B} \) and \(\mathcal{B} ^{\prime} \):
    \[
        \mathcal{B} = \left\{ (1,-1),(0,1) \right\} \qquad \mathcal{B} ^{\prime} = \left\{ (1,1,0),(0,1,1),(1,0,1) \right\} 
    \]
\end{exercise}
\begin{longsolution}
    First, we must take \(T\) on the basis \(\mathcal{B} \). Then we must express it in terms of the basis \(\mathcal{B} ^{\prime} \). Consider the first element:
    \[
        T(1,-1)=(0,1,-1)
    \]
    Let there exist scalars \(a,b,c\) such that \((0,1,-1) = a(1,1,0)+b(0,1,1)+c(1,0,1)\) and define the coordinates of \(T(1,-1)\) relative to \(\mathcal{B} ^{\prime} \) to be 
    \[
        [T(1,-1)]_{\mathcal{B} ^{\prime} }= \begin{bmatrix}[r]
             a \\
             b \\
             c \\
        \end{bmatrix}
    \]
    Constructing a system of equations gives 
    \[
        \begin{bmatrix}[r]
            1 &0  &1  &0   \\
             1&1  &0  &1   \\
             0&1  &1  &-1   \\
        \end{bmatrix}\xlongrightarrow{rref} \begin{bmatrix}[r]
            1 & 0 &0  &1   \\
             0&1  &0  & 0  \\
             0&0  &1  & -1  \\
        \end{bmatrix}
    \]
    Thus we have \([T(1,-1)]_{\mathcal{B} ^{\prime} }=(1,0,-1)\). Now consider the second basis element:
    \[
        T(0,1)=(1,0,1)
    \]
    This is the third element of \(\mathcal{B} ^{\prime} \), so we give \([T(0,1)]_{\mathcal{B} ^{\prime} }=(0,0,1)\). We can now construct \(A\). From \ref{thismakessense},
    \[
        A = \begin{bmatrix}[r]
            1 &0   \\
             0&0   \\
             -1&1   \\
        \end{bmatrix}
    \]
\end{longsolution}
\begin{exercise}
    Using the above exercise, find \(T(\mathbf{v})\) where \(\mathbf{v}=(5,4)\) 
\end{exercise}
\begin{solution}
    First find \([\mathbf{v}]_\mathcal{B} \). Since the first element is the only one without \(0\) in the first entry, trivially we have \([\mathbf{v}]_\mathcal{B} =(5,9)\). It follows from the definition of \(A\) that 
    \[
        [T(\mathbf{v})]_{\mathcal{B} ^{\prime} } = \begin{bmatrix}[r]
            1 &0   \\
             0&0   \\
             -1&1   \\
        \end{bmatrix} \begin{bmatrix}[r]
             5 \\
             9 \\
        \end{bmatrix} = \begin{bmatrix}[r]
             5 \\
             0 \\
             4 \\
        \end{bmatrix}
    \]
    Converting back to the standard basis, we have 
    \[
        \mathbf{v}= 5(1,1,0) + 4(1,0,1) = (9,5,4)
    \]
    Thus, \(T(\mathbf{v}) = (9,5,4)\). We can verify this by recalling \(T(x,y)=(x+y,x,y)\), and seeing that 
    \[
        T(5,4) = (5+4,5,4)=(9,5,4)
    \]
\end{solution}
\section{Transition Matrices and Similarity}
Suppose we know some matrix \(A\) that gives a linear map \(T\) relative to some basis \(\mathcal{B} \), but we want the matrix \(A^{\prime} \) that gives \(T\) relative to a different basis \(\mathcal{B} ^{\prime} \). Let \(P\) give \([\mathbf{v}]_\mathcal{B} = P[\mathbf{v}]_{\mathcal{B} ^{\prime} }\). It follows that \([\mathbf{v}]_{\mathcal{B} ^{\prime} }=P ^{-1} [\mathbf{v}]_\mathcal{B} \). By taking \(P[\mathbf{v}]_{\mathcal{B} ^{\prime} }\), we can convert \(\mathbf{v}\) into the basis \(\mathcal{B} \) and then use \(A\) to perform the transformation \(T\):
\[
    [T(\mathbf{v})]_{\mathcal{B} } = AP[\mathbf{v}]_{\mathcal{B} ^{\prime} }
\]
Taking \(P ^{-1} \) on this \([T(\mathbf{v})]_{\mathcal{B} }\) gives 
\[
    [T(\mathbf{v})]_{\mathcal{B}^{\prime}  } = P ^{-1} AP [\mathbf{v}]_{\mathcal{B} ^{\prime} }
\]
This is the idea of similar matrices.
\begin{definition}[Similar Matrices]
    For \(A,A^{\prime} \in M_{n,n}\), we say that \(A^{\prime} \) is similar to \(A\) if there exists some invertible matrix \(P\in M_{n,n} \) such that 
    \[
        A^{\prime} =P ^{-1} A P
    \]
\end{definition}
\begin{theorem}
    Similar matrices define an equivalence relation on the set of all matrices. That is, if \(\sim \) denotes similar matrices, then
    \[
        A\sim A
    \] 
    \[
        A\sim B \Longrightarrow B\sim A
    \]
    \[
        A\sim B \land B\sim C \Longrightarrow A\sim C
    \]
    for all \(A,B,C\in M_{n,n}\).
\end{theorem}
\begin{proof}
    Let \(A,B,C\in M_{n,n}\). Trivially, \(A\sim A\) by the identity matrix:
    \[
        I_nAI_n^{-1} =A
    \]
    \[
        \therefore A\sim A
    \]
    Now let \(A\sim B\). Thus, there exists some invertible \(P\in M_{n,n}\) such that 
    \[
        A = P ^{-1} B P
    \]
    Since \(P\) is invertible, we have 
    \begin{align*}
        \Longrightarrow&   PA = BP\\
        \Longrightarrow& PA P ^{-1} =B
    \end{align*}
    \[
        \therefore A\sim B \Longrightarrow B\sim A
    \]
    Finally, let \(A\sim B\) and \(B\sim C\). Thus, there exist invertible matrices \(P,L\in M_{n,n}\) such that
    \[
        A = P ^{-1} BP
    \]
    and
    \[
        B = L^{-1} C L
    \]
    It follows that we have 
    \[
        A=P ^{-1} L^{-1} CLP
    \]
    By \ref{prowotogen}, \((LP)^{-1} =P ^{-1} L^{-1} \). Define \(M\coloneqq LP\) and \(M^{-1} \coloneqq (LP)^{-1} =P ^{-1} L^{-1} \). We thus have 
    \[
        A = M^{-1} CM
    \]
    Since \(M\in M_{n,n}\), we are done. Therefore, 
    \[
        A\sim B\land B\sim C\Longrightarrow A\sim C
    \]
    and thus \(\sim \) defines an equivalence relation on \(M_{n,n}\).
\end{proof}
One advantage of using similar matrices, which we will see on full display in the next chapter, is finding diagonal matrices that act on a set equivalently to a different matrix, since the products of diagonal matrices are much easier to compute.