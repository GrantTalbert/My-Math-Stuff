%Pulling basic packages
\usepackage[T1]{fontenc} %Better font encoding
\usepackage[margin=0.75in]{geometry} %page size formatting
\usepackage{graphicx} %colors
\graphicspath{C:/Users/83tal/Documents/School/Multivariable Calculus (MAT-201)/Figures/}

%Math
\usepackage{amsmath, amsfonts, mathtools, amsthm, amssymb}
\usepackage{cancel} %Lines in environmental environment denoting canceled terms
\usepackage{systeme} %Curly bracket aligning of systems of equations
\usepackage{mathrsfs}
\newcommand\mat[1]{\mathbf{#1}}


\usepackage[usenames,dvipsnames]{xcolor}
\usepackage[hyphens]{url}
\usepackage[style=alphabetic,maxcitenames=2]{biblatex} %bibliography
\usepackage[colorlinks=true,linkcolor=cyan,urlcolor=green,citecolor=red]{hyperref}
\usepackage{float}
\usepackage{booktabs} %nicer looking tables
\usepackage{emptypage}
\usepackage{subcaption}

%Better lists
\usepackage{enumitem}

%Multicolumn support
\usepackage{multicol}
\setlength\multicolsep{0pt}

\usepackage{caption}
\captionsetup{belowskip=0pt}

% for the big braces
\usepackage{bigdelim}

%SI units
\usepackage{siunitx}

%Building listings environment for code
\usepackage{listings}
\definecolor{backcolour}{rgb}{0.21,0.29,0.31}

%tikz aka witchcraft
\usepackage{tikz}
\usetikzlibrary{intersections, angles, quotes, positioning}
\usetikzlibrary{arrows.meta}
\usepackage{pgfplots}
\pgfplotsset{compat=1.18}


\tikzset{
	force/.style={thick, {Circle[length=2pt]}-stealth, shorten <=-1pt}
}

% Algorithm environment which i never expect to use
\usepackage[linesnumbered,lined,vlined,ruled,commentsnumbered,resetcount,algochapter]{algorithm2e}
\SetKwComment{Comment}{// }{}
\SetArgSty{textsl}
\def\algocflineautorefname{Algorithm}

\makeatletter
\def\itemautorefname{\@gobble}
\makeatother

%Theorem environments
\makeatother
\usepackage{thmtools}
\usepackage[framemethod=TikZ]{mdframed}

\mdfsetup{skipabove=1em,skipbelow=0em}

\theoremstyle{definition}

\declaretheoremstyle[
headfont=\bfseries\sffamily\color{ForestGreen!70!black}, bodyfont=\normalfont,
mdframed={
	linewidth=2pt,
	rightline=false, topline=false, bottomline=false,
	linecolor=ForestGreen, backgroundcolor=ForestGreen!5,
	nobreak=false
}
]{thmgreenbox}

\declaretheoremstyle[
headfont=\bfseries\sffamily\color{ForestGreen!70!black}, bodyfont=\normalfont,
mdframed={
	linewidth=2pt,
	rightline=false, topline=false, bottomline=false,
	linecolor=ForestGreen, backgroundcolor=ForestGreen!8,
	nobreak=false
}
]{thmgreen2box}

\declaretheoremstyle[
headfont=\bfseries\sffamily\color{NavyBlue!70!black}, bodyfont=\normalfont,
mdframed={
	linewidth=2pt,
	rightline=false, topline=false, bottomline=false,
	linecolor=NavyBlue, backgroundcolor=NavyBlue!5,
	nobreak=false
}
]{thmbluebox}

\declaretheoremstyle[
headfont=\bfseries\sffamily\color{TealBlue!70!black}, bodyfont=\normalfont,
mdframed={
	linewidth=2pt,
	rightline=false, topline=false, bottomline=false,
	linecolor=TealBlue, backgroundcolor=TealBlue!5,
	nobreak=false
}
]{thmblueline}

\declaretheoremstyle[
headfont=\bfseries\sffamily\color{RawSienna!70!black}, bodyfont=\normalfont,
mdframed={
	linewidth=2pt,
	rightline=false, topline=false, bottomline=false,
	linecolor=RawSienna, backgroundcolor=RawSienna!5,
	nobreak=false
}
]{thmredbox}

\declaretheoremstyle[
headfont=\bfseries\sffamily\color{RawSienna!70!black}, bodyfont=\normalfont,
mdframed={
	linewidth=2pt,
	rightline=false, topline=false, bottomline=false,
	linecolor=RawSienna, backgroundcolor=RawSienna!8,
	nobreak=false
}
]{thmred2box}

\declaretheoremstyle[
headfont=\bfseries\sffamily\color{SeaGreen!70!black}, bodyfont=\normalfont,
mdframed={
	linewidth=2pt,
	rightline=false, topline=false, bottomline=false,
	linecolor=SeaGreen, backgroundcolor=SeaGreen!2,
	nobreak=false
}
]{thmgreen3box}

\declaretheoremstyle[
headfont=\bfseries\sffamily\color{WildStrawberry!70!black}, bodyfont=\normalfont,
mdframed={
	linewidth=2pt,
	rightline=false, topline=false, bottomline=false,
	linecolor=WildStrawberry, backgroundcolor=WildStrawberry!5,
	nobreak=false
}
]{thmpinkbox}

\declaretheoremstyle[
headfont=\bfseries\sffamily\color{MidnightBlue!70!black}, bodyfont=\normalfont,
mdframed={
	linewidth=2pt,
	rightline=false, topline=false, bottomline=false,
	linecolor=MidnightBlue, backgroundcolor=MidnightBlue!5,
	nobreak=false
}
]{thmblue2box}

\declaretheoremstyle[
headfont=\bfseries\sffamily\color{Gray!70!black}, bodyfont=\normalfont,
mdframed={
	linewidth=2pt,
	rightline=false, topline=false, bottomline=false,
	linecolor=Gray, backgroundcolor=Gray!5,
	nobreak=false
}
]{notgraybox}

\declaretheoremstyle[
headfont=\bfseries\sffamily\color{Gray!70!black}, bodyfont=\normalfont,
mdframed={
	linewidth=2pt,
	rightline=false, topline=false, bottomline=false,
	linecolor=Gray,
	nobreak=false
}
]{notgrayline}

% \declaretheoremstyle[
% 	headfont=\bfseries\sffamily\color{RawSienna!70!black}, bodyfont=\normalfont,
% 	numbered=no,
% 	mdframed={
	% 			linewidth=2pt,
	% 			rightline=false, topline=false, bottomline=false,
	% 			linecolor=RawSienna, backgroundcolor=RawSienna!1,
	% 		},
% 	qed=\qedsymbol
% ]{thmproofbox}

\declaretheoremstyle[
headfont=\bfseries\sffamily\color{NavyBlue!70!black}, bodyfont=\normalfont,
numbered=no,
mdframed={
	linewidth=2pt,
	rightline=false, topline=false, bottomline=false,
	linecolor=NavyBlue, backgroundcolor=NavyBlue!1,
	nobreak=false
}
]{thmexplanationbox}

\declaretheoremstyle[
headfont=\bfseries\sffamily\color{WildStrawberry!70!black}, bodyfont=\normalfont,
numbered=no,
mdframed={
	linewidth=2pt,
	rightline=false, topline=false, bottomline=false,
	linecolor=WildStrawberry, backgroundcolor=WildStrawberry!1,
	nobreak=false
}
]{thmanswerbox}

\declaretheoremstyle[
headfont=\bfseries\sffamily\color{Violet!70!black}, bodyfont=\normalfont,
mdframed={
	linewidth=2pt,
	rightline=false, topline=false, bottomline=false,
	linecolor=Violet, backgroundcolor=Violet!1,
	nobreak=false
}
]{conjpurplebox}

\declaretheorem[style=thmgreenbox, name=Definition, numberwithin=section]{definition}
\declaretheorem[style=thmgreen2box, name=Definition, numbered=no]{definition*}
\declaretheorem[style=thmredbox, name=Theorem, numberwithin=section]{theorem}
\declaretheorem[style=thmred2box, name=Theorem, numbered=no]{theorem*}
\declaretheorem[style=thmredbox, name=Lemma, numberwithin=section]{lemma}
\declaretheorem[style=thmredbox, name=Proposition, numberwithin=section]{proposition}
\declaretheorem[style=thmredbox, name=Corollary, numberwithin=section]{corollary}
\declaretheorem[style=thmpinkbox, name=Problem, numberwithin=section]{problem}
\declaretheorem[style=thmpinkbox, name=Problem, numbered=no]{problem*}
\declaretheorem[style=thmblue2box, name=Claim, numbered=no]{claim}
\declaretheorem[style=conjpurplebox, name=Conjecture, numberwithin=section]{conjecture}

% Redefine proof environment to get a full control. 
\makeatletter
\renewenvironment{proof}[1][\proofname]{\par
	\pushQED{\qed}%
	\normalfont \topsep-2\p@\@plus6\p@\relax
	\trivlist
	\item[\hskip\labelsep
	\color{RawSienna!70!black}\sffamily\bfseries
	#1\@addpunct{.}]\ignorespaces
	\begin{mdframed}[linewidth=2pt,rightline=false, topline=false, bottomline=false,linecolor=RawSienna, backgroundcolor=RawSienna!3]
	}{%
		\popQED\endtrivlist\@endpefalse
	\end{mdframed}
}
\makeatother

\declaretheorem[style=thmbluebox, numbered=no, name=Example]{eg}
\declaretheorem[style=thmexplanationbox, numbered=no, name=Proof]{tmpexplanation}
\newenvironment{explanation}[1][]{\vspace{-10pt}\pushQED{\(\circledast\)}\begin{tmpexplanation}}{\null\hfill\popQED\end{tmpexplanation}}

\declaretheorem[style=thmblueline, numbered=no, name=Remark]{remark}
\declaretheorem[style=thmblueline, numbered=no, name=Note]{note}
\declaretheorem[style=thmbluebox, numbered=no, name=Exercise]{exercise}
\declaretheorem[style=notgrayline, numbered=no, name=As previously seen]{prev}
\declaretheorem[style=thmgreen3box, numbered=no, name=Intuition]{intuition}
\declaretheorem[style=notgraybox, numbered=no, name=Notation]{notation}
\declaretheorem[style=thmexplanationbox, numbered=no, name=Answer]{answer}

\usepackage{etoolbox}
\renewcommand{\qed}{\null\hfill\(\blacksquare\)}

\makeatletter

\def\testdateparts#1{\dateparts#1\relax}
\def\dateparts#1 #2 #3 #4 #5\relax{
	\marginpar{\small\textsf{\mbox{#1 #2 #3 #5}}}
}

\def\@lecture{}%
\newcommand{\lecture}[3]{
	\ifthenelse{\isempty{#3}}{%
		\def\@lecture{Lecture #1}%
	}{%
		\def\@lecture{Lecture #1: #3}%
	}%
	\section*{\@lecture}
	\marginpar{\small\textsf{\mbox{#2}}}
}
\usepackage{pgffor}%
\newcommand{\lec}[2]{%
	\foreach \c in {#1,...,#2}{%
		\IfFileExists{Lectures/lec_\c.tex} {%
			\input{Lectures/lec_\c.tex}%
		}{}%
	}%
}

% fancy headers
\usepackage{fancyhdr}
\pagestyle{fancy}

% LE: left even
% RO: right odd
% CE, CO: center even, center odd
\fancyhead[LE,RO]{Grant Talbert}

\fancyhead[RO,LE]{\@lecture} % Right odd,  Left even
\fancyhead[RE,LO]{}          % Right even, Left odd
\fancyfoot[RO,LE]{\thepage}  % Right odd,  Left even
\fancyfoot[RE,LO]{}          % Right even, Left odd
\fancyfoot[C]{\leftmark}     % Center

\makeatother

% notes
\usepackage[color=cyan]{todonotes}
\usepackage{marginnote}
\let\marginpar\marginnote

% Fix some stuff
% %http://tex.stackexchange.com/questions/76273/multiple-pdfs-with-page-group-included-in-a-single-page-warning
\pdfsuppresswarningpagegroup=1

% Appendix environment
\usepackage{appendix}
\def\chapterautorefname{Section}
\def\sectionautorefname{Section}
\def\appendixautorefname{Appendix}
\renewcommand\appendixname{Appendix}
\renewcommand\appendixtocname{Appendix}
\renewcommand\appendixpagename{Appendix}
% begin appendix autoref patch [\autoref subsections in appendix](https://tex.stackexchange.com/questions/149807/autoref-subsections-in-appendix)
\makeatletter
\patchcmd{\hyper@makecurrent}{%
	\ifx\Hy@param\Hy@chapterstring
	\let\Hy@param\Hy@chapapp
	\fi
}{%
	\iftoggle{inappendix}{%true-branch
		% list the names of all sectioning counters here
		\@checkappendixparam{chapter}%
		\@checkappendixparam{section}%
		\@checkappendixparam{subsection}%
		\@checkappendixparam{subsubsection}%
		\@checkappendixparam{paragraph}%
		\@checkappendixparam{subparagraph}%
	}{}%
}{}{\errmessage{failed to patch}}

\newcommand*{\@checkappendixparam}[1]{%
	\def\@checkappendixparamtmp{#1}%
	\ifx\Hy@param\@checkappendixparamtmp
	\let\Hy@param\Hy@appendixstring
	\fi
}
\makeatletter

\newtoggle{inappendix}
\togglefalse{inappendix}

\apptocmd{\appendix}{\toggletrue{inappendix}}{}{\errmessage{failed to patch}}
\apptocmd{\subappendices}{\toggletrue{inappendix}}{}{\errmessage{failed to patch}}
% end appendix autoref patch

\setcounter{tocdepth}{3}

\lstdefinestyle{code}{
	backgroundcolor=\color{backcolour},
	commentstyle=\color{LimeGreen},
	keywordstyle=\color{Cyan},
	numberstyle=\tiny\color{LimeGreen},
	stringstyle=\color{BrickRed},
	basicstyle=\ttfamily\footnotesize,
	breakatwhitespace=false,         
	breaklines=true,                 
	captionpos=b,                    
	keepspaces=true,                 
	numbers=left,                    
	numbersep=5pt,                  
	showspaces=false,                
	showstringspaces=false,
	showtabs=false,                  
	tabsize=2
}
\lstset{style=code}


% My commands
\newcommand{\R}{\mathbb{R}} %Real numbers
\newcommand{\C}{\mathbb{C}} %Complex numbers
\newcommand{\HS}{\mathscr{H}} %Preferred notation for Hilbert Spaces
\DeclarePairedDelimiter\bra{\langle}{\rvert} %Bra
\DeclarePairedDelimiter\ket{\lvert}{\rangle} %Ket
\DeclarePairedDelimiterX\braket[2]{\langle}{\rangle}{#1\,\delimsize\vert\,\mathopen{}#2} %Bra-ket
\newcommand{\ihat}{\boldsymbol{\hat{\textbf{\i}}}}
\newcommand{\jhat}{\boldsymbol{\hat{\textbf{\j}}}}
\newcommand{\khat}{\boldsymbol{\hat{\textbf{k}}}}
\newcommand{\hati}{\boldsymbol{\hat{\textbf{\i}}}}
\newcommand{\hatj}{\boldsymbol{\hat{\textbf{\j}}}}
\newcommand{\hatk}{\boldsymbol{\hat{\textbf{k}}}}
\newcommand{\pvec}[1]{\vec{#1}\mkern2mu\vphantom{#1}} % from https://tex.stackexchange.com/questions/120029/how-to-typeset-a-primed-vector