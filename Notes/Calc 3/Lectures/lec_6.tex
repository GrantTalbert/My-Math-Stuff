\begin{exercise}
    Evaluate \(\iint\limits_{R}\left( 3x+4y^2 \right)dA \), where \(R\) is the region bdd by \(x^2 +y^2 =4\) and \(x^2+y^2 =9\), where \(y\geq 0\).
\end{exercise}
\begin{solution}
    \[
        R =\left\{ (r\cos \theta ,r\sin \theta )\mid \theta \in[0,\pi ]\land r\in[2,3] \right\}
    \]
    \begin{align*}
        \iint\limits_{R}\left( 3x+4y^2 \right)dA&=\int_0^\pi \int_2^3 \left( 3r\cos \theta +4(r\sin \theta )^2 \right)r\,drd \theta \\
        &=\int_0^\pi \int_2^3 3r^2 \cos \theta +4r^{3}\sin^2 \theta \,drd \theta \\
        &=\int_0^\pi r^3 \cos \theta +r^4 \sin^2 \theta \biggl\vert_2^3 \,d \theta \\
        &=\int_0^\pi 19\cos \theta +65\sin^2 \theta \,d \theta \\
        &=10\sin \theta \big\vert_0^\pi +\int \frac{65}{2}\left( 1-\cos 2\theta  \right) \,d \theta \\
        &=\frac{65}{2}\left( \theta -\frac{\sin 2\theta }{2} \right) \biggl\vert_0^\pi =\frac{65\pi}{2}
    \end{align*}
\end{solution}
\begin{exercise}
    \[
        \int _{-5}^5 \int_0^{\sqrt{25-x^2} } \left( \frac{1}{2}x^ 2+y^2 \right)\,dydx
    \]
\end{exercise}
\begin{solution}
    \begin{align*}
        \int _{-5}^5 \int_0^{\sqrt{25-x^2} } \left( \frac{1}{2}x^ 2+y^2 \right)\,dydx&= \int_0^\pi \int_0^5 \frac{1}{2}r^3 \cos^2 \theta + r^3 \sin ^2 \theta \,drd \theta\\
        &=\int_0^\pi \frac{r^4}{8}\cos ^2\theta +\frac{r^4}{4}\sin ^2\theta \biggl\vert_0^5\,d \theta \\
        &=\int_0^\pi \frac{625}{8}\cos ^2\theta +\frac{625}{4}\sin ^2\theta \,d \theta \\
        &=\frac{625}{8}\int_0^\pi \cos ^2\theta +2\sin ^2\theta \,d \theta \\
        &= \frac{625}{8}\int_0^\pi \cos ^2\theta +\sin ^2\theta +\sin ^2\theta \,d \theta \\
        &=\frac{625}{8}\int_0^\pi 1+\sin ^2\theta \,d \theta \\
        &=\frac{625}{16}\int_0^\pi 1-\cos 2\theta \,d \theta \\
        &=\frac{625\theta}{16}-\frac{625\sin2\theta}{32}\biggl\vert_0^\pi \\
        &=\frac{625\pi}{16}
    \end{align*}
\end{solution}
\section{Applications of Double Integrals}
A lamina is basically an extremely thin film, so integrals over regions can approximate properties of a lamina. 
\begin{definition}[Center of Mass of a Lamina]\label{hi}
    Let \(\rho :D \to \mathbb{R}\) be the density function of a lamina defined over some region \(D\). The mass of the lamina is given as 
    \[
        m=\iint\limits_{D}\rho (x,y)\,dA
    \]
    and the center of mass of the lamina is given as 
    \[
        (\overline{x} ,\overline{y} )=\left( \frac{M_x}{m},\frac{M_y}{m} \right)
    \]
    where 
    \[
        M_x =\iint\limits_{D}y \rho (x,y)\,dA\qquad M_y =\iint\limits_{D}x \rho (x,y)\,dA
    \]
\end{definition}
\begin{remark}
    This definition of mass has applications beyond just a lamina. For example, given a charge density function \(\sigma (x,y)\) defined over a region \(D\), then the total charge \(Q\) is calculated similarly.
\end{remark}
\begin{definition}[Moment of a Lamina about an Axis]
    The moment of a lamina defined over the region \(D\) with density function \(z=\rho (x,y)\) about the \(x\) or \(y\) axis is given as 
    \[
        M_x =\iint\limits_{D} y \rho (x,y)\,dA\qquad M_y =\iint\limits_{D}x \rho (x,y)\,dA
    \]
    Notice the use of moment in \ref{hi}
\end{definition}
\begin{definition}[Moment of Inertia]
    The moment of inertia of a lamina about an axis is given as 
    \[
        I_x =\iint\limits_{D}y^2 \rho (x,y)\,dA\qquad I_y =\iint\limits_{D}x^2 \rho (x,y)\,dA
    \]
\end{definition}
The moments about an axis measure the tendency of a lamin ato rotate about an axis, and the moments of inertia measure the difficulty required to start/stop such a rotation.\\
Another application of double integrals concerns joint probability functions. Let \(f\) be probability function for some variable \(X\). Thus, \(f\) is a function such that 
\begin{enumerate}
    \item \(f(x)\geq 0\,\forall x\in\mathbb{R}\)
    \item \(\int_\mathbb{R} f(x)\,dx =1\) 
\end{enumerate}
The probability that \(X\) lies between \(a,b\in\mathbb{R}\) is given as 
\[
    P(a\leq X\leq b)=\int_a^b f(X)\,dX
\]
However, we can use calculus to extend this to two variables.
\begin{definition}[Joint Density Function]
    Let \(X,Y\in\mathbb{R}\) be probabilistic variables with a probability density \(f(x,y)\). The probability that \(X\in[a,b]\) and \(Y\in[c,d]\) is given as 
    \[
        f(a\leq X\leq b,c\leq Y\leq d)=\int_a^b \int_c^d f(X,Y)\,dYdX
    \]
\end{definition}
\section{Surface Area}
\begin{definition}[Surface Area]
    Let \(f:D\subset \mathbb{R}^2\to \mathbb{R}\) and let \(f_x,f_y\) be continuous on \(D\). The surface area of the surface \(z=f(x,y)\) is given as
    \[
        SA=\iint\limits_{D}\sqrt{1+f_x ^2 +f_y ^2} 
    \]
\end{definition}
    This was a long section.
\section{Triple Integrals}
\begin{definition}[Triple Integral]
    Let \(B \subset \mathbb{R}^3\) with \(f:B\to \mathbb{R}\). We give 
    \[
        \iiint\limits_{B}f(x,y,z)\,dV \coloneqq \lim_{\ell,n,m \to \infty}\sum_{i=1}^\ell \sum_{j=1}^n \sum_{k=1}^m f \left( x_i^*,y_j^*,z_k^* \right) \Delta x \Delta y \Delta z
    \]
\end{definition}
\ref{Fubini} holds for triple integrals, and the integral will always exist if the function is continuous over the region of integration.\\
Here are triple integral analogues of some of the applications that were seen earlier.
\begin{definition}[Mass of an Object]\label{huge}
    Let an object occupy some region \(E \subseteq \mathbb{R}^3\) with density function \(\rho :E\to \mathbb{R}\). The mass of the object is given as 
    \[
        m=\iiint\limits_{E} \rho (x,y,z)\,dV
    \]
\end{definition}
\begin{definition}[Center of Mass of an Object]
    Consider the definitions in \ref{huge}. The center of mass of an object is defined as 
    \[
        \left( \overline{x} ,\overline{y} ,\overline{z}  \right) = \left( \frac{M_{yz} }{m},\frac{M_{xz} }{m},\frac{M_{xy} }{m} \right)
    \]
    where we give
    \[
        M_{yz}=\iiint\limits_{E}x \rho (x,y,z)\,dV \qquad M_{xz}=\iiint\limits_{E}y \rho (x,y,z)\,dV \qquad M_{xy}=\iiint\limits_{E}z \rho (x,y,z)\,dV
    \]
\end{definition}
\begin{exercise}
    Evaluate \(\iiint\limits_{B}\sqrt{x^2 +z^2}\,dV \) where \(B\) is bdd by \(y=x^2 +z^2\) and \(y=4\).
\end{exercise}
\begin{solution}
    Recall how to integrate in polar coordinates for this problem. We will convert the \(xz\)-axis into polar coordinates.
    \begin{align*}
        \iiint\limits_{B}\sqrt{x^2 +z^2}  &= \int_{-2}^2 \int_{-\sqrt{4-x^2} }^{\sqrt{4-x^2} }\int_{x^2 +z^2}^4\, dydzdx\\
        &= \int_0^{2\pi }\int_0^2 \int_{r^2 \cos^2\theta +r^{2}\sin ^2\theta  }^4 \sqrt{r^2 \cos ^2 \theta +r^2 \sin ^2\theta } \,dy(rdrd \theta )\\
        &=\int_0^{2\pi }\int_0^2 \int_{r^2}^4 r^2 \,drd \theta \\
        &=\int_0^{2\pi }\int_0^2 \left( 4-r^2 \right)r^2\,drd \theta \\
        &=\int_0^{2\pi }\int_0^2 4r^2 -r^4\\
        &=\int_0^{2\pi } \frac{4r^3}{3}-\frac{r^5}{5}\biggl\vert_0^2\,d \theta \\
        &=\int_0^{2\pi }\frac{32}{3}-\frac{32}{5}\,d \theta \\
        &=2\pi \left( \frac{32}{3}-\frac{32}{5} \right) 
    \end{align*}
\end{solution}
\begin{theorem}
    For some arbitrary region \(E \subset \mathbb{R}^3\), the volume can be calculated by evaluating 
    \[
        \iiint\limits_{E}\,dV
    \]
\end{theorem}
\begin{exercise}
    Find the volume of the region \(B\) which is bdd by \(x^2 +y^2 =9\), \(y+z=5\), and \(z=1\). After integrating wrt \(z\), convert to polar coordinates.
\end{exercise}
\begin{solution}
    \begin{align*}
        V&=\iiint\limits_{B}\,dV\\
        &=\int_{-3}^3 \int_{-\sqrt{9-x^2} }^{\sqrt{9-x^2} }\int_1^{5-y}\,dzdydx\\
        &=\int_{-3}^3 \int_{-\sqrt{9-x^2} }^{\sqrt{9-x^2} } 4-y\,dydx\\
        &=\int_{0}^{2\pi }\int_0^3 4r-r^2\sin \theta \,drd \theta \\
        &=\int_0^{2\pi } \left[2r^2 -\frac{1}{3}r^3 \sin \theta \right]_0^3 \,d \theta \\
        &=\int_0^{2\pi }18-9\sin \theta \,d \theta \\
        &=36\pi
    \end{align*}
\end{solution}
We just integrated in cylindrical coordinates!
\section{Triple Integrals in Cylindrical Coordinates}
An intuitive way to think about cylindrical coordinates I've found is to think of polar coordinates along an entire orthogonal axis.\\
To convert between cylindrical and cartesian coordinates, we give 
\[
    x=r\cos \theta \qquad y=r\sin \theta \qquad z=z
\]
To convert from cartesian to cylindrical coordinates, we give
\[
    r^2 =x^2 +y^2 \qquad \tan \theta =\frac{y}{x}\qquad z=z
\]
\begin{exercise}
    Express the volume of the object bdd by \(z=4-x^2 -y^2\), \(x^2 +y^2 =4\), and \(z=9+x^2 +y^2\) as a triple integral in cartesian coordinates, then as a double integral in cartesian coordinates, then as a triple integral in cylindrical coordinates, then as a double integral in polar coordinates.
\end{exercise}
\begin{solution}
    Notice \(9+x^2 +y^2 \geq 4-x^2 -y^2 \,\forall x,y\in\mathbb{R} \) because \(\forall (x\in\mathbb{R})\left( x^2 \geq 0 \right) \).
    \[
        \int_{-2}^2 \int_{-\sqrt{4-x^2} }^{\sqrt{4-x^2} }\int_{4-x^2 -y^2}^{9+x^2 +y^2}\,dzdydx
    \]
    \[
        \int_{-2}^2 \int_{-\sqrt{4-x^2} }^{\sqrt{4-x^2} } 5 +2x^2 +2y^2 \,dydx
    \]
    \[
        \int_{0}^{2\pi }\int_0^2 \int_{4-r^2}^{9+r^2}r\,dzdrd \theta
    \]
    \[
        \int_{0}^{2\pi }\int_0^2 \left( 5+2r^2 \right)r\,drd \theta 
    \]
\end{solution}
The above question will probably be on the exam.
\section{Triple Integrals in Spherical Coordinates}
Spherical coordinates are given by the ordered triple \(\rho ,\theta ,\phi \), where 
\begin{itemize}
    \item \(\rho \) denotes the distance from the origin
    \item \(\theta \) denotes the angle formed with the positive \(x\)-axis
    \item \(\phi \) denotes the angle formed with the positive \(z\)-axis
\end{itemize}
We require that
\[
    \rho \geq 0\qquad \phi \in[0,\pi ]
\]
To convert between spherical to cartesian coordinates, use 
\[
    x=\rho \sin \phi \cos \theta \qquad x=\rho \sin \phi \sin \theta \qquad z=\rho \cos \phi 
\]
and
\[
    \rho ^2=x^2 +y^2 +z^2
\]
\begin{intuition}
    This is like adding \(\rho \sin \phi \) to the polar coordinates for \(x\) and \(y\), and this leaves \(\cos \phi \) ``left over'', so we let \(\rho \cos \phi \) be \(z\).
\end{intuition}
\begin{theorem}[Integrals in Spherical Coordinates]
    Let \(E \subset \mathbb{R}^3\) with \(f:E\to \mathbb{R}\). Let \(E\) be defined in spherical coordinates as 
    \[
        E\coloneqq \left\{ (\rho ,\theta ,\phi )\mid \rho \in[a,b]\land \theta \in[\alpha ,\beta ]\land \phi \in[c,d] \right\} 
    \]
    where \(a\geq 0\), \(\beta -\alpha \leq 2\pi \), \(d-c\leq \pi \), and \(c\geq 0\). We give 
    \[
        \iiint\limits_{E}f(x,y,z)\,dV =\int_c^d \int_\alpha^\beta \int_a^b f \left( \rho  \cos \theta \sin \phi , \rho \sin \theta \sin \phi ,\rho \cos \phi \right) \rho ^2 \sin \phi \,d \rho d \theta d \phi 
    \]
\end{theorem}
\begin{exercise}
    Convert the following integral to spherical coordinates:
    \[
        \int_{-1}^1 \int_0^{\sqrt{1-y^2} }\int_{-\sqrt{1-x^2 -y^2} }^{\sqrt{1-x^2 -y^2} } f(x,y,z)\,dzdxdy
    \]
\end{exercise}
\begin{solution}
    Notice that this defines a unit semicircle where \(x\geq 0\). Thus, we can give 
    \[
        \int_0^\pi \int_{-\frac{\pi}{2}}^{\frac{\pi}{2}}\int_0^1 f \left( \rho  \cos \theta \sin \phi , \rho \sin \theta \sin \phi ,\rho \cos \phi \right) \rho ^2 \sin \phi \,d \rho d \theta d \phi 
    \]
\end{solution}