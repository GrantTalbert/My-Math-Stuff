%basic packages
\usepackage[utf8]{inputenc}
\usepackage[T1]{fontenc}
\usepackage{textcomp}
\usepackage{graphicx}
\usepackage[margin=0.75in]{geometry}
\usepackage[usenames,dvipsnames]{xcolor}

%math
\usepackage{amsmath, amsthm, amsfonts, amssymb, mathtools}
\usepackage{mathrsfs}
\usepackage{cancel}
\usepackage{siunitx} %phyjsicsssss

%misc
\usepackage{float}
\usepackage[hyphens]{url}
\usepackage{darkmode}
\enabledarkmode
\definecolor{page}{HTML}{293133}

\usepackage{hyperref}
\hypersetup{
    colorlinks,
    linkcolor={cyan},
    urlcolor={green},
    citecolor={red}
}

%my commands
\DeclarePairedDelimiter\bra{\langle}{\rvert} %Bra
\DeclarePairedDelimiter\ket{\lvert}{\rangle} %Ket
\DeclarePairedDelimiterX\braket[2]{\langle}{\rangle}{#1\,\delimsize\vert\,\mathopen{}#2} %Bra-ket
\newcommand{\pvec}[1]{\vec{#1}\mkern2mu\vphantom{#1}} % from https://tex.stackexchange.com/questions/120029/how-to-typeset-a-primed-vector
\newcommand{\hati}{\boldsymbol{\hat{\textbf{\i}}}}
\newcommand{\hatj}{\boldsymbol{\hat{\textbf{\j}}}}
\newcommand{\hatk}{\boldsymbol{\hat{\textbf{k}}}}

%theorems
\usepackage{thmtools}
\usepackage{tikz}
\usepackage{tikz-cd}
\usepackage[framemethod=TikZ]{mdframed}
\mdfsetup{skipabove=1em,skipbelow=0em, innertopmargin=5pt, innerbottommargin=6pt}
\theoremstyle{definition} %because obviously

%theorem styles
\declaretheoremstyle[headfont=\bfseries\sffamily\color{white}, bodyfont=\normalfont\color{white}, mdframed={nobreak, backgroundcolor=page, linecolor=white}]{thmbox} %makes a box
\declaretheoremstyle[headfont=\bfseries\sffamily\color{white}, bodyfont=\normalfont\color{white}, mdframed={nobreak, backgroundcolor=page, linecolor=white, bottomline=false, topline=false, rightline=false}, qed=\(\blacksquare\)]{proofline} %makes a line on the right with a qed
\declaretheoremstyle[headfont=\bfseries\sffamily\color{white}, bodyfont=\normalfont\color{white}, mdframed={nobreak, backgroundcolor=page, linecolor=white, bottomline=false, topline=false, rightline=false}, qed=\qedsymbol]{egline} %makes a line on the right with a qed
\declaretheoremstyle[headfont=\bfseries\sffamily\color{white}, bodyfont=\normalfont\color{white}]{nobox} %doesnt make a box

%numbered theorems
\declaretheorem[numberwithin=chapter, style=thmbox, name=Definition]{definition}
\declaretheorem[sibling=definition, style=thmbox, name=Theorem]{theorem}
\declaretheorem[sibling=definition, style=thmbox, name=Lemma]{lemma}
\declaretheorem[sibling=definition, style=thmbox, name=Corollary]{corollary}
\declaretheorem[sibling=definition, style=thmbox, name=Proposition]{proposition}

%unnumbered theorems
\declaretheorem[numbered=no, style=nobox, name=Remark]{remark}
\declaretheorem[numbered=no, style=nobox, name=Exercise]{exercise}
\declaretheorem[numbered=no, style=nobox, name=Notation]{notation}
\declaretheorem[numbered=no, style=nobox, name=Note]{note}
\declaretheorem[numbered=no, style=nobox, name=As Previously Seen]{prev}
\declaretheorem[numbered=no, style=nobox, name=Intuition]{intuition}
\declaretheorem[numbered=no, style=nobox, name=Example]{eg}

%solution environment
\declaretheorem[numbered=no, style=egline, name=Solution]{setsolution}
\newenvironment{solution}[1][]{\vspace{-10pt}\begin{setsolution}}{\end{setsolution}}
\newenvironment{answer}[1][]{\vspace{-10pt}\begin{setsolution}}{\end{setsolution}}
\newenvironment{explanation}[1][]{\vspace{-10pt}\begin{setsolution}}{\end{setsolution}}
%proof environment
\declaretheorem[numbered=no, style=proofline, name=Proof]{replacementproof}
\renewenvironment{proof}[1][\proofname]{\begin{replacementproof}}{\end{replacementproof}}
%\renewenvironment{proof}[1][\proofname]{\vspace{-12pt}\begin{replacementproof}}{\end{replacementproof}} this version has no space between above and proof section

%pulls lecture files
\newcommand{\lec}[2]{%
	\foreach \c in {#1,...,#2}{%
		\IfFileExists{Lectures/lec_\c.tex} {%
			\input{Lectures/lec_\c.tex}%
		}{}%
	}%
}
%to use, in the same file directory as your header.tex and master.tex files, create a folder titled "Lectures" and put your
%lectures into their, named lec_1.tex, lec_2.tex, and so on. In the master.tex file, write \lec{a}{b} where a is the lowest
%number you want to call, and b is the highest.

%fancy headers
\usepackage{fancyhdr}
\pagestyle{fancy}
\fancyhead{}\fancyfoot{}
\fancyfoot[R]{\thepage}
\fancyfoot[C]{\leftmark}


%figures, taken from (https://castel.dev/post/lecture-notes-2)
\usepackage{import}
\usepackage{xifthen}
\usepackage{pdfpages}
\usepackage{transparent}
\newcommand{\incfig}[1]{%
    \def\svgwidth{\columnwidth}
    \import{./figures/}{#1.pdf_tex}
}

\author{Grant Talbert}