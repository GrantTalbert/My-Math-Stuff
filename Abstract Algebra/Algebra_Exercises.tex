\documentclass{article}
\usepackage{amsmath}
\usepackage{amsfonts, amssymb,   mathtools, mathrsfs, amsthm}
\usepackage{color, soul}
\usepackage{mdframed}
\usepackage{graphicx}
\usepackage[margin=1in]{geometry}
\theoremstyle{definition}
\newtheorem{environment}{Exercise}
\newenvironment{exercise}
    {\begin{mdframed}\begin{environment}}
    {\end{environment}\end{mdframed}}
\begin{document}
\section{Preliminiaries}
\begin{exercise}
    Suppose that
\[
    A=\left\{ x\mid x\in\mathbb{N} \land x\text{ is even}   \right\}
\]
\[
    B=\left\{ x\mid x\in\mathbb{N} \land x\text{ is prime}   \right\}
\]
\[
    C=\left\{ x\mid x\in\mathbb{N} \land x\text{ is a multiple of }5   \right\}
\]
Describe the following sets:
\[
    (a)\quad A\cap B \qquad \qquad\qquad (c)\quad A\cup B
\]
\[
    \hspace{1cm}(b)\quad B\cap C \qquad\qquad\qquad (d)\quad A \cap (B \cup  C)
\]
\end{exercise}
The only prime number that isnt odd is \(2\), so \(A\cap B = \{ 2 \} \).\\
A prime number is, by definition, a number that is only a multiple of itself and 1. As such, the only multiple of 5 that can be prime is 5 itself, since the only multiples of 5 are 5 and 1. Thus, \(B \cap  C = \{ 5 \} \).\\
I'm not fully sure how to describe \(A \cup  B\) other than as ``the set of all prime numbers and all even numbers: \(\left\{ x\in\mathbb{N} \mid \frac{x}{2}\in\mathbb{N} \lor x\text{ is prime}  \right\} \) ''\\
\(A \cap (B \cup  C) = (A \cap B) \cup  (A \cap  C)\). We've already sesen that \(A \cap B = \{ 2 \} \). We also have \(A \cap C\) is every multiple of \(5\) divisible by 2. This is equivalent to every multiple of \(5\cdot2\), which equals 10. Thus, \(A \cap C = \{ x\mid x\in\mathbb{N} \land x\text{ is a multiple of }10  \} \). Notice that \(2\) is not a multiple of 10, so \(2\notin A \cap  C\). Thus, \((A \cap  B)\cup (A\cap C)=\{ x\mid x\in\mathbb{N} \land (x\text{ is a multiple of 10}\lor x=2 ) \} \).\\
\begin{exercise}
    If \(A=\{ a,b,c \} \), \(B=\{ 1,2,3 \} \), \(C=\{ x \} \), and \(D=\varnothing \), list all elements in each of the following sets.
\begin{align*}
    &(a)\quad A\times B &&(c)\quad A\times B\times C\\
    &(b)\quad B\times A &&(d)\quad A\times D
\end{align*}
\end{exercise}
\[
    A\times B=\{ (a,1),(a,2),(a,3),(b,1),(b,2),(b,3),(c,1),(c,2),(c,3) \} 
\]
\[
    B\times A =\{ (1,a),(1,b),(1,c),(2,a),(2,b),(2,c),(3,a),(3,b),(3,c) \} 
\]
\[
    A\times B\times C = \{ (a,1,x),(a,2,x),(a,3,x),(b,1,x),(b,2,x),(b,3,x),(c,1,x),(c,2,x),(c,3,x) \} 
\]
\[
    A\times D=\varnothing 
\]
\begin{exercise}
    Find an example of two nonempty sets \(A,B\) for which \(A\times B=B\times A\) is true.
\end{exercise}
Let \(A=\{ 1,2 \} \) and \(B=\{ 1,2 \} \). Then we have
\[
    A\times B = \{(1,1),(1,2),(2,1),(2,2)\}
\]
\[
    B\times A=\{ (1,1),(1,2),(2,1),(2,2) \} 
\]
\begin{exercise}
    Prove \(A \cup \varnothing =A\) and \(A\cap \varnothing =\varnothing \).
\end{exercise}
By definition, there does not exist any \(x\in\varnothing \). It follows directly from this fact that there does not exist any \(x\in\varnothing \) such that \(x\in A\). Now consider the following.
\[
    A \cap \varnothing =\left\{ x\mid x\in A \land x\in \varnothing  \right\} 
\]
However, we just stated that there exists no such \(x\) satisfying the requirements of this set. Therefore, the set must be empty. Thus,
\[
   \therefore A \cap \varnothing =\varnothing 
\]
Similarly, we have
\[
    A \cup  \varnothing =\{ x\mid x\in a \lor x\in \varnothing  \} 
\]
Since there is no \(x\in \varnothing \), but there may be some \(x\in A\), it follows that the only elements of \(A \cup \varnothing \) are elements of \(A\), since there are no elements in \(\varnothing \). 
\[
    \therefore A \cup \varnothing =A
\]
\begin{exercise}
    Prove \(A\cup B=B\cup A\) and \(A\cap B=B\cap A\)
\end{exercise}
By definition, \(x\in A \cup B\) if and only if \(x\in A\) or \(x\in B\). This is equivalent to saying \(x\in B\) or \(x\in A\), thus \(A \cup B=B\cup A\). Similarly, some \(x\in A\cap B\) must be in \(A\) and \(B\). This is equivalent to \(x\in B\) and \(x\in A\), thus \(A \cap B=B\cap A\).
\begin{exercise}
    Prove \(A \cup (B\cap C)=(A\cup B)\cap (A \cup C)\)     
\end{exercise}
Some \(x\in A\cup (B\cap C)\) must be either \(x\in B\) and \(x\in C\), or it must satisfy \(x\in A\). If \(x\in A\), then \(x\in A\cup B\) and \(x\in A \cup C\). Therefore, \(x\in (A\cup B)\cap (A\cup C)\). Suppose \(x\notin A\) but \(x \in B\cap C\). Then \(x\in A\cup B\) and \(x\in A\cup C\), so equivalently, \(x\in (A\cup B)\cap (A\cup C)\). Now suppose \(x\notin A\cup (B\cap C)\). It must be true that \(x\notin A\), since \(x\in A \Longrightarrow x\in A\cup (B\cap C)\). However, it's not implied that \(x\notin B\) or \(x\notin C\), only that \(x\) cannot be an element of \emph{both} sets. Suppose \(x\in B\). The same logic employed here will work for \(x\in C\). It follows that \(x\in A\cup B\), but \(x\notin A\cup C\). Therefore, \(x\notin (A\cup B)\cap (A\cup C)\). For some \(x\) that is not an element of any of the sets, trivially \(x\notin (A\cup B)\cap (A\cup C)\) and \(x\notin A\cup (B\cap C)\). Therefore, \(A\cup (B\cap C)=(A\cup B)\cap (A\cup C)\).
\begin{exercise}
    Prove \(A \cap (B\cup C)=(A\cap B)\cup (A \cap C)\)     
\end{exercise}
Some \(x\in A\cap (B\cup C)\) must satisfy \(x\in A\) and either \(x\in B\) or \(x\in C\). Thus, any \(x\notin A\Longrightarrow x\notin A\cap (B\cup C)\). Similarly, any \(x\notin A\) will satisfy \(x\notin A\cap B\) and \(x\notin A\cap C\). Thus, \(x\notin (A\cap B)\cup (A\cap C)\). Similarly, any \(x\notin B\) and \(x\notin C\) will satisfy both \(x\notin A\cap B\) and \(x\notin A\cap C\). Thus, \(x\notin (A\cap B)\cup (A\cap C)\). Trivially, any \(x\notin A\), \(x\notin B\), and \(x\notin C\) will satisfy \(x\notin (A\cap B)\cup (A\cap C)\). Finally, some \(x\in A\) and either \(x\in B\), \(x\in C\), or \(x\in B\) and \(x\in C\), will satisfy either \(x\in A\cap B\), \(x\in A\cap C\), or both. Thus, \(x\in (A\cap B)\cup (A\cap C)\). Therefore, we see that \(A\cap (B\cup C)=(A\cap B)\cup (A\cap C)\).
\begin{exercise}
    Prove \(A \subset B\) if and only if \(A\cap B=A\).
\end{exercise}
I don't believe this textbook made any distinction between the symbols \(\subset \) and \(\subseteq \), so I will use \(\subset \) as I would typically use \(\subseteq \) for this problem set. If \(A \subset B\), then every single element of \(A\) must also be an element of \(B\). If every element of \(A\) is an element of \(B\) as well, then \(A\cap B\) must contain every element of \(A\). However, \(A \cap B\) cannot contain more elements than \(A\), since then there would be at least one element in \(B\) that is not in \(A\). Therefore, \(A \subset B \Longrightarrow  A\cap B = A\).\\
Conversely, if \(A \cap B = A\), then every element of \(A\) must also be an element of \(B\). By definition of a subset, this implies that \(A \subset B\). Therefore, \(A \cap B = A \Longrightarrow A \subset B\).\\
\[
    \therefore A \subset B \iff  A\cap B=A
\]
\begin{exercise}
    Prove \((A\cap B)^{\prime} =A^{\prime} \cup B^{\prime} \) 
\end{exercise}
By definition, \(A^{\prime} \) is the set of all things not in \(A\) that are in the \emph{universal set} that we happen to be working under. As such, \((A\cap B)^{\prime} \) is the set of all things that are not in both \(A\) and \(B\) at the same time. Any \(x\notin A\) will satisfy \(x\notin A\cap B\), and any \(x\notin B\) will satisfy \(x\notin A\cap B\) as well. This is equivalent to saying \(x\in A^{\prime} \) or \(x\in B^{\prime} \) implies \(x\notin (A\cap B)^{\prime} \), or \(x\in A^{\prime} \cup B^{\prime} \Longrightarrow x\notin (A\cap B)^{\prime} \).
\begin{exercise}
    Prove \(A \cup B=(A \cap B)\cup (A\setminus B)\cup (B\setminus A)\) 
\end{exercise}
\(A\cup B\) is the set of all things in either \(A\) or \(B\). As such, any \(x\in A\cap B\), that is anything in both \(A\) and \(B\), will also have \(x\in A\cup B\). Furthermore, \((A \cup B)\setminus (A \cap B)\) will give the set of all things in either \(B\) but not in \(A\), or the set of all things in either \(A\) but not in \(B\). This translates to the set \(A\setminus B\) and the set \(B\setminus A\). Thus we have \((A\cup B)\setminus (A\cap B)=(A\setminus B)\cup (B\setminus A)\). We also know that since \(A\cap B\) is the set of all things in both \(A\) and \(B\), it must be true that \(x\in A\cap B \Longrightarrow x\in A\cup B\). Thus, \((A \cap B)\subset (A \cup B)\). If some \(X \subset Y\), then \((Y\setminus X)\cup X = Y\), because the set \(Y\setminus X\) is the set of all \(x\in Y\) with \(x\notin X\), but the union of this set with \(X\) gives the set of all \(x\in Y\) and \(x\in X\). And since \(X \subset Y\), there are no elements introduced that were not originally in \(Y\). This implies that \(((A\cup B)\setminus (A\cap B))\cup (A\cap B)=A\cup B\), since everything in \(A\cap B\) is also in \(A\cup B\). Recall \((A\cup B)\setminus (A\cap B)=(A\setminus B)\cup (B\setminus A)\). Since \(A=A \Longrightarrow A\cup B = A\cup B\), we can say
\[
    (A\cup B)\setminus (A\cap B)\cup (A\cap B)=(A\setminus B)\cup (B\setminus A)\cup (A\cap B)
\]
\[
    \Longrightarrow A\cup B=(A\cap B)\cup (A\setminus B)\cup (B\setminus A)
\]
\begin{exercise}
    Prove \((A\cup B)\times C = (A\times C)\cup (B\times C)\)
\end{exercise}
We know that
\[
    (A\cup B)\times C = \left\{ (x,y)\mid x\in A \cup B \land y\in C \right\} 
\]
Since the second element will always be some \(y\in C\) but the first element will be in either \(A\) or \(B\), we can break this up into
\[
    \left\{ (x,y)\mid x\in A \cup B \land y\in C \right\} =\left\{ (x,y)\mid x\in A \land y\in C \right\} \cup \left\{ (x,y)\mid x\in B \land y\in C \right\} =(A\times C)\cup (B\times C)
\]
\begin{exercise}
    Prove \((A\cap B)\setminus B=\varnothing \) 
\end{exercise}
We know that \(x\in A\cap B\) if and only if \(x\in A\) and \(x\in B\). Thus, for all \(x\in A\cap B\), it follows that \(x\in B\). Additionally, \((A\cap B)\setminus B\) is the set of all \(x\in A\cap B\) that satisfy \(x\notin B\). However, all \(x\in A\cap B\) must satisfy \(x\in B\), and thus cannot satsify \(x\notin B\) by the definition of a set. It follows that \((A\cap B)\setminus B\) must be empty, that is, \((A\cap B)\setminus B=\varnothing \).
\begin{exercise}
    Prove \((A\cup B)\setminus B=A\setminus B\).
\end{exercise}
Some \(x\in A\cup B\) must satisfy either \(x\in A\), \(x\in B\), or both. It follows that some \(x\in (A\cup B)\setminus B\) must satisfy the initial properties required for \(x\in A\cup B\), but must also satisfy \(x\notin B\). The only \(x\notin B\) with \(x\in A\cup B\) are the \(x\in A\) with \(x\notin B\). The set of all \(x\in A\) where \(x\notin B\) is equivalent to \(A\setminus B\). Thus, \((A\cup B)\setminus B=A\setminus B\).
\begin{exercise}
    Prove \(A\setminus (B\cup C)=(A\setminus B)\cap (A\setminus C)\).
\end{exercise}
The set \(A\setminus (B\cup C)=\left\{ x\mid x\in A \land x\notin B\cup C  \right\} \). The statement \(x\notin B\cup C\) is true if and only if \(x\notin B\) and \(x\notin C\), since \(x\in B\cup C\) if \(x\in B\) or \(x\in C\). This is equivalent to \(x\in B^{\prime} \) and \(x\in C^{\prime} \), or \(x\in B^{\prime} \cap C^{\prime} \). Thus, we have
\[
A\setminus (B\cup C)=\left\{ x\mid x\in A \land x\in B^{\prime} \cap  C^{\prime}  \right\}=\left\{ x\mid x\in A \land x\in B^{\prime}  \right\} \cap \left\{ x\mid x\in A \land x\in C^{\prime}  \right\}=(A\setminus B)\cap (A\setminus C)
\]
\begin{exercise}
    Prove \(A\cap (B\setminus C)=(A\cap B)\setminus (A\cap C)\).
\end{exercise}
\(x\in A\cap (B\setminus C)\) iff \(x\in A\), \(x\in B\), and \(x\notin C\). As such, we have 
\[
    A\cap (B\setminus C)=A\cap B\cap C^{\prime} =(A\cap B)\setminus C
\]
However, if \(x\in C \Longrightarrow x\notin A\cap (B\setminus C)\), and if \(x\in A\cap C \Longrightarrow x\in A\) and \(x\in C\), and if \(x\in A\cap (B\setminus C)\Longrightarrow x\in A\), the above statement is equivalent to
\[
    (A\cap B)\setminus (A\cap C)
\]
\begin{exercise}
    Prove \((A\setminus B)\cup (B\setminus A)=(A\cup B)\setminus (A\cap B)\).
\end{exercise}
\[
    (A\setminus B)\cup (B\setminus A)=(A\cap B^{\prime} )\cup (B\cap A^{\prime} )=((A\cap B^{\prime} )\cup B)\cap ((A\cap B^{\prime} )\cup A^{\prime} )=((A\cup B)\cap (B^{\prime} \cup B))\cap ((A^{\prime}\cup A )\cap (A^{\prime} \cup B^{\prime} ))
\]
\[
    =(A\cup B)\cap (A^{\prime} \cup B^{\prime} )=(A\cup B)\setminus (A^{\prime} \cup B^{\prime} )^{\prime} =(A\cup B)\setminus (A\cap B)
\]
\begin{exercise}
    Which of the following relations \(f:\mathbb{Q} \to \mathbb{Q} \) define a mapping? In each case, supply a reason why \(f\) is or is not a mapping.
    \begin{align*}
        &(a)\quad f(p/q)=\frac{p+1}{p-2} &&(c)\quad f(p/q)=\frac{p+q}{q^2}\\
        &(b)\quad f(p/q)=\frac{3p}{3q} &&(d)\quad f(p/q)=\frac{3p^2}{7q^2}-\frac{p}{q}
    \end{align*}
\end{exercise}
(a) is not a mapping since equivalent inputs can give different outputs. For example, notice \(\frac{1}{2}=\frac{4}{8}\). Consider
\[
    f(1/2)=\frac{1+1}{1-2}=-2
\]
However,
\[
    f(4/8)=\frac{4+1}{4-2}=\frac{5}{3}\neq -2
\]
Therefore, (a) cannot be a mapping.\\
(b) is a mapping. Notice that
\[
    f(p/q)=\frac{3p}{3q}=\frac{3}{3}\frac{p}{q}=\frac{p}{q}
\]
\[
    \Longrightarrow f(p/q)=\frac{p}{q}
\]
Therefore, any equivalent value of \(p/q\) will have a well defined map.\\
(c) is not a mapping. Consider \(\frac{1}{3}=\frac{3}{9}\). We have
\[
    f(1/3)=\frac{1+3}{3^2}=\frac{4}{9}
\]
However,
\[
    f(3/9)=\frac{3+9}{9^2}=\frac{12}{81}=\frac{1}{12}\neq \frac{4}{9}
\]
Therefore, (c) is not well defined and cannot be a map.\\
(d) is a map. Recall that \(f(p/q)=\frac{p}{q}\) is a map. We have
\[
    f(p/q)=\frac{3p^2}{7q^2}-\frac{p}{q}=\frac{3}{7}\frac{p}{q}\frac{p}{q}-\frac{p}{q}
\]
Any time the variables are used, there is no ambiguity in their value due each fraction basically representing the identity function.
\begin{exercise}
    Determine which of the following functions are one-to-one (injective) and which are onto (surjective). If the function is not onto, determine its range.
    \begin{align*}
        &(a)\quad f:\mathbb{R} \to \mathbb{R} \text{ defined by }f(x)=e^x\\
        &(b)\quad f:\mathbb{Z} \to \mathbb{Z} \text{ defined by }f(n)=n^2 + 3\\
        &(c)\quad f:\mathbb{R} \to \mathbb{R} \text{ defined by }f(x)=\sin x\\
        &(d)\quad f:\mathbb{Z} \to \mathbb{Z} \text{ defined by }f(x)=x^2    
    \end{align*}
\end{exercise}
(a) is an injective but not surjective function. To prove injectivity, suppose \(e^x = e^y\). It follows that \(\ln \left( e^x \right)=\ln \left( e^{y}  \right) \Longrightarrow x=y \). Therefore \(f\) is injective. However, there does not exist any \(x\in \mathbb{R} \) with \(e^x \leq 0\), so the function is not surjective. The range of the function is \(\mathbb{R} ^+\).\\
(b) is not injective nor surjective. For any \(n\in\mathbb{Z} \), it follows that \(n^2 \geq 0\). As such, \(n^2 + 3 \geq 3\), and thus the range of the function is \(\{ x\in\mathbb{Z} \mid x\geq 3 \} \subsetneq  \mathbb{Z} \). Furthermore, for any \(n\in\mathbb{Z} \), \(f(n)=f(-n)\) since \(f(-n)=(-n)^2 + 3 = (-1)^2(n)^2 + 3=1n^2 + 3=n^2 + 3=f(n)\). Therefore, the function is not injective either.\\
(c) is also not injective or surjective. For all \(x\in\mathbb{R} \), \(\sin \) is bounded by \(\sin x\in [-1,1] \subsetneq \mathbb{R} \). Thus, \(f\) is not surjective, and the range of \(f\) is \([-1,1]\). \(f\) is also not injective, since values differing by a factor of \(2\pi \) give the same \(\sin x\) value. For example, \(f(2\pi )=\sin (2\pi )=0=\sin (0)=f(0)\). Therefore \(f(0)=f(2\pi )\) and \(f\) is not injective.\\
(d) is ALSO not injective nor surjective. Again, \(f(-x)=f(x)\) since \(f(-x)=(-x)^2=(-1)^2(x)^2=1x^2=x^2=f(x)\) for any \(x\in\mathbb{Z} \), so \(f\) is not injective. Furthermore, for any \(x\in\mathbb{Z} \), we have \(x^2 \geq 0\). Therefore, the range of \(f\) is \(\mathbb{N}\), where \(0\in\mathbb{N} \), and thus \(f\) is not surjective.

\end{document}